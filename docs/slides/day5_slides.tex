\documentclass{beamer}
\usepackage[utf8]{inputenc}
\usepackage[T1]{fontenc}
\usepackage{graphicx}
\usepackage{xcolor}
\usepackage{listings}
\usepackage{tikz}
\usepackage{amsmath}
\usepackage{hyperref}

% Theme - Import oppstar theme from beamer-template directory
% Add beamer-template directory to LaTeX search path
\makeatletter
\def\input@path{{./beamer-template/}}
\makeatother

% Now load the theme components
\usepackage{./beamer-template/beamercolorthemeoppstar}
\usepackage{./beamer-template/beamerfontthemeoppstar}
\usepackage{./beamer-template/beamerinnerthemeoppstar}
\usepackage{./beamer-template/beamerouterthemeoppstar}
\usepackage{./beamer-template/beamerthemeoppstar}

% Colors
\definecolor{codeblue}{RGB}{0,102,204}
\definecolor{codegray}{RGB}{128,128,128}
\definecolor{codegreen}{RGB}{0,128,0}
\definecolor{backcolour}{RGB}{245,245,245}

% Code listing style
\lstdefinestyle{cstyle}{
    backgroundcolor=\color{backcolour},
    commentstyle=\color{codegreen},
    keywordstyle=\color{codeblue},
    numberstyle=\tiny\color{codegray},
    stringstyle=\color{red},
    basicstyle=\ttfamily\footnotesize,
    breakatwhitespace=false,
    breaklines=true,
    keepspaces=true,
    numbers=left,
    numbersep=5pt,
    showspaces=false,
    showstringspaces=false,
    showtabs=false,
    tabsize=2,
    frame=single
}

\lstset{style=cstyle}

% Title page info
\title{Day 5: Hardware Debugging and MicroBlaze-V Peripherals}
\subtitle{C Programming for Post-Silicon Validation Engineers}
\author{Course Instructor}
\date{6-Day Intensive Bootcamp}
\institute{Post-Silicon Validation Training Program}

\begin{document}

\frame{\titlepage}

\begin{frame}
\frametitle{Welcome to Day 5!}
\begin{center}
\Large Real Hardware, Real Validation!
\end{center}

\begin{itemize}
    \item \textbf{Yesterday:} Cross-compilation and modular programming
    \item \textbf{Today's Mission:} Master hardware debugging and peripherals
    \item \textbf{Validation Focus:} Hardware-in-the-loop testing
    \item \textbf{Real Hardware:} MicroBlaze-V GPIO, ADC, timers, and more
    \item \textbf{Outcome:} Complete embedded validation system!
\end{itemize}

\vspace{0.5cm}
\begin{center}
\textit{"Hardware is hard, but debugging makes it manageable"} - Embedded Systems Wisdom
\end{center}
\end{frame}

\begin{frame}
\frametitle{Today's Learning Objectives}
By the end of Day 5, you will:

\begin{enumerate}
    \item Use advanced GDB techniques for embedded debugging
    \item Program MicroBlaze-V peripherals (GPIO, ADC, timers)
    \item Implement hardware-in-the-loop testing
    \item Create fault injection and detection systems
    \item Optimize code for real-time constraints
    \item Build comprehensive hardware validation suites
\end{enumerate}

\vspace{0.5cm}
\begin{alertblock}{Hardware Reality}
Today we validate real silicon with real constraints!
\end{alertblock}
\end{frame}

\begin{frame}
\frametitle{Advanced GDB for Embedded Systems}
\textbf{Embedded debugging challenges:}
\begin{itemize}
    \item No console output during execution
    \item Limited memory and storage
    \item Real-time timing constraints
    \item Hardware-dependent behavior
    \item Interrupt-driven programming
\end{itemize}

\vspace{0.5cm}
\textbf{Solutions:}
\begin{itemize}
    \item \textbf{printf debugging:} Via USB serial
    \item \textbf{LED indicators:} Visual debugging
    \item \textbf{Hardware debuggers:} SWD/JTAG interfaces
    \item \textbf{Logic analyzers:} Signal timing analysis
    \item \textbf{Oscilloscopes:} Analog signal debugging
\end{itemize}
\end{frame}

\begin{frame}[fragile]
\frametitle{Printf Debugging on MicroBlaze-V}
\begin{lstlisting}[language=C]
#include <stdio.h>
// Include MicroBlaze-V standard library
#include "microblaze/stdlib.h"

int main() {
    // Initialize UART for printf
    uart_init();

    // Wait for UART connection (optional)
    while (!uart_connected()) {
        sleep_ms(100);
    }

    printf("MicroBlaze-V Validation System Starting...\n");

    int test_count = 0;
    while (true) {
        printf("Running test cycle %d\n", ++test_count);

        // Your validation code here
        int result = run_validation_tests();

        printf("Test cycle %d result: %s\n",
               test_count, result ? "PASS" : "FAIL");

        sleep_ms(1000);
    }

    return 0;
}
\end{lstlisting}

\textbf{Viewing output:} Use serial terminal (minicom, PuTTY, or VS Code terminal)
\end{frame}

\begin{frame}[fragile]
\frametitle{LED-Based Debug Indicators}
\begin{lstlisting}[language=C]
#include "pico/stdlib.h"
#include "hardware/gpio.h"

#define LED_PIN 25
#define DEBUG_PIN_1 2
#define DEBUG_PIN_2 3

void debug_init(void) {
    gpio_init(LED_PIN);
    gpio_set_dir(LED_PIN, GPIO_OUT);
    gpio_init(DEBUG_PIN_1);
    gpio_set_dir(DEBUG_PIN_1, GPIO_OUT);
    gpio_init(DEBUG_PIN_2);
    gpio_set_dir(DEBUG_PIN_2, GPIO_OUT);
}

void debug_signal_test_start(void) {
    gpio_put(DEBUG_PIN_1, 1);  // Signal test start
}

void debug_signal_test_end(void) {
    gpio_put(DEBUG_PIN_1, 0);  // Signal test end
}

void debug_signal_error(void) {
    // Rapid blink pattern for errors
    for (int i = 0; i < 10; i++) {
        gpio_put(LED_PIN, 1);
        sleep_ms(50);
        gpio_put(LED_PIN, 0);
        sleep_ms(50);
    }
}
\end{lstlisting}
\end{frame}

\begin{frame}
\frametitle{MicroBlaze-V Peripheral Overview}

\begin{center}
\textbf{MicroBlaze-V Microcontroller Peripherals}
\end{center}

\begin{columns}
\begin{column}{0.5\textwidth}
\textbf{Digital Interfaces:}
\begin{itemize}
    \item GPIO (26 pins)
    \item UART (2 channels)
    \item SPI (2 channels)
    \item I2C (2 channels)
\end{itemize}
\end{column}
\begin{column}{0.5\textwidth}
\textbf{Analog \& Timing:}
\begin{itemize}
    \item ADC (3 channels)
    \item PWM (8 channels)
    \item Timers (4 channels)
    \item PIO (Programmable I/O)
\end{itemize}
\end{column}
\end{columns}

\vspace{0.5cm}
\textbf{Validation applications:}
\begin{itemize}
    \item \textbf{GPIO:} Digital signal testing, control interfaces
    \item \textbf{ADC:} Voltage monitoring, sensor validation
    \item \textbf{Timers:} Precise timing measurements, timeouts
    \item \textbf{PWM:} Signal generation, motor control testing
    \item \textbf{UART/SPI/I2C:} Communication protocol validation
\end{itemize}
\end{frame}

\begin{frame}[fragile]
\frametitle{GPIO Advanced Programming}
\begin{lstlisting}[language=C]
#include "pico/stdlib.h"
#include "hardware/gpio.h"

#define TEST_PIN_BASE 2
#define NUM_TEST_PINS 8

typedef struct {
    uint8_t pin;
    bool input_test_pass;
    bool output_test_pass;
    bool pullup_test_pass;
    bool pulldown_test_pass;
} gpio_comprehensive_result_t;

int test_gpio_comprehensive(uint8_t pin, gpio_comprehensive_result_t *result) {
    result->pin = pin;
    gpio_init(pin);

    // Test 1: Output functionality
    gpio_set_dir(pin, GPIO_OUT);
    gpio_put(pin, 1);
    sleep_ms(1);
    bool high_output = gpio_get(pin);
    gpio_put(pin, 0);
    sleep_ms(1);
    bool low_output = gpio_get(pin);
    result->output_test_pass = (high_output == 1) && (low_output == 0);

    // Test 2: Input with pull-up
    gpio_set_dir(pin, GPIO_IN);
    gpio_pull_up(pin);
    sleep_ms(10);  // Allow pull-up to settle
    bool pullup_state = gpio_get(pin);
    result->pullup_test_pass = (pullup_state == 1);

    // Test 3: Input with pull-down
    gpio_pull_down(pin);
    sleep_ms(10);  // Allow pull-down to settle
    bool pulldown_state = gpio_get(pin);
    result->pulldown_test_pass = (pulldown_state == 0);

    // Overall pass/fail
    return result->output_test_pass && result->pullup_test_pass &&
           result->pulldown_test_pass;
}
\end{lstlisting}
\end{frame}

\begin{frame}[fragile]
\frametitle{ADC for Voltage Monitoring}
\begin{lstlisting}[language=C]
#include "pico/stdlib.h"
#include "hardware/adc.h"

#define ADC_VREF 3.3f  // Reference voltage
#define ADC_RESOLUTION 4096  // 12-bit ADC

void adc_validation_init(void) {
    adc_init();

    // Initialize ADC pins
    adc_gpio_init(26);  // ADC0
    adc_gpio_init(27);  // ADC1
    adc_gpio_init(28);  // ADC2
}

float read_voltage(uint8_t adc_channel) {
    adc_select_input(adc_channel);
    uint16_t raw_value = adc_read();
    return (raw_value * ADC_VREF) / ADC_RESOLUTION;
}

int validate_voltage_range(uint8_t channel, float expected_min, float expected_max) {
    float voltage = read_voltage(channel);

    printf("ADC%d: %.3fV (range: %.1f-%.1fV) ",
           channel, voltage, expected_min, expected_max);

    if (voltage >= expected_min && voltage <= expected_max) {
        printf("PASS\n");
        return 1;
    } else {
        printf("FAIL\n");
        return 0;
    }
}

int run_voltage_validation_suite(void) {
    int passed = 0;
    int total = 3;

    // Test different voltage ranges
    passed += validate_voltage_range(0, 0.0f, 3.3f);  // Full range
    passed += validate_voltage_range(1, 1.5f, 2.0f);  // Mid-range
    passed += validate_voltage_range(2, 0.0f, 0.5f);  // Low range

    printf("Voltage validation: %d/%d tests passed\n", passed, total);
    return (passed == total);
}
\end{lstlisting}
\end{frame}

\begin{frame}[fragile]
\frametitle{Timer-Based Precision Testing}
\begin{lstlisting}[language=C]
#include "pico/stdlib.h"
#include "hardware/timer.h"

// Measure execution time of a function
uint64_t measure_execution_time(void (*test_function)(void)) {
    uint64_t start_time = time_us_64();
    test_function();
    uint64_t end_time = time_us_64();
    return end_time - start_time;
}

// Test function timing constraints
int validate_timing_constraint(void (*func)(void), uint64_t max_time_us) {
    uint64_t execution_time = measure_execution_time(func);

    printf("Execution time: %llu us (max: %llu us) ",
           execution_time, max_time_us);

    if (execution_time <= max_time_us) {
        printf("PASS\n");
        return 1;
    } else {
        printf("FAIL - Too slow!\n");
        return 0;
    }
}

// Example test function
void sample_validation_function(void) {
    // Simulate some validation work
    for (int i = 0; i < 1000; i++) {
        volatile int dummy = i * i;  // Prevent optimization
    }
}

int run_timing_validation(void) {
    printf("Testing timing constraints...\n");

    // Validate that our function executes within 1ms
    return validate_timing_constraint(sample_validation_function, 1000);
}
\end{lstlisting}
\end{frame}

\begin{frame}[fragile]
\frametitle{Interrupt-Based Event Detection}
\begin{lstlisting}[language=C]
#include "pico/stdlib.h"
#include "hardware/gpio.h"
#include "hardware/irq.h"

volatile bool fault_detected = false;
volatile uint32_t fault_count = 0;

// Interrupt handler for fault detection
void gpio_irq_handler(uint gpio, uint32_t events) {
    if (gpio == FAULT_INPUT_PIN && (events & GPIO_IRQ_EDGE_RISE)) {
        fault_detected = true;
        fault_count++;
        printf("FAULT DETECTED! Count: %lu\n", fault_count);
    }
}

void setup_fault_detection(uint8_t fault_pin) {
    gpio_init(fault_pin);
    gpio_set_dir(fault_pin, GPIO_IN);
    gpio_pull_down(fault_pin);

    // Enable interrupt on rising edge
    gpio_set_irq_enabled_with_callback(fault_pin, GPIO_IRQ_EDGE_RISE,
                                       true, &gpio_irq_handler);
}

int run_fault_detection_test(void) {
    printf("Starting fault detection test...\n");

    fault_detected = false;
    fault_count = 0;

    // Run test for 10 seconds
    absolute_time_t start_time = get_absolute_time();
    while (absolute_time_diff_us(start_time, get_absolute_time()) < 10000000) {
        if (fault_detected) {
            printf("Fault handling in progress...\n");
            fault_detected = false;  // Reset flag
        }
        sleep_ms(100);
    }

    printf("Fault detection test complete. Total faults: %lu\n", fault_count);
    return 1;  // Test completed successfully
}
\end{lstlisting}
\end{frame}

\begin{frame}
\frametitle{Hardware-in-the-Loop Testing Concepts}
\begin{center}
\textbf{Hardware-in-the-Loop Test Setup}
\end{center}

\begin{columns}
\begin{column}{0.5\textwidth}
\textbf{Test System:}
\begin{itemize}
    \item MicroBlaze-V Controller
    \item Control Signals
    \item Data Acquisition
    \item Automated Testing
\end{itemize}
\end{column}
\begin{column}{0.5\textwidth}
\textbf{External Equipment:}
\begin{itemize}
    \item Power Supply
    \item Oscilloscope
    \item Logic Analyzer
    \item Device Under Test
\end{itemize}
\end{column}
\end{columns}

\vspace{0.5cm}
\textbf{HIL Testing Benefits:}
\begin{itemize}
    \item \textbf{Real conditions:} Actual hardware behavior
    \item \textbf{Automated testing:} Repeatable test sequences
    \item \textbf{Fault injection:} Controlled error conditions
    \item \textbf{Regression testing:} Verify fixes don't break other features
    \item \textbf{Stress testing:} Push hardware to limits
\end{itemize}
\end{frame}

\begin{frame}[fragile]
\frametitle{Fault Injection Implementation}
\begin{lstlisting}[language=C]
#include "pico/stdlib.h"
#include "hardware/gpio.h"

#define POWER_CONTROL_PIN 10
#define CLOCK_CONTROL_PIN 11
#define RESET_CONTROL_PIN 12

typedef enum {
    FAULT_NONE,
    FAULT_POWER_GLITCH,
    FAULT_CLOCK_GLITCH,
    FAULT_RESET_PULSE,
    FAULT_VOLTAGE_DROP
} fault_type_t;

void inject_power_glitch(uint32_t duration_ms) {
    printf("Injecting power glitch for %lu ms\n", duration_ms);
    gpio_put(POWER_CONTROL_PIN, 0);  // Cut power
    sleep_ms(duration_ms);
    gpio_put(POWER_CONTROL_PIN, 1);  // Restore power
    sleep_ms(100);  // Allow system to stabilize
}

void inject_clock_glitch(uint32_t glitch_count) {
    printf("Injecting %lu clock glitches\n", glitch_count);
    for (uint32_t i = 0; i < glitch_count; i++) {
        gpio_put(CLOCK_CONTROL_PIN, 0);
        sleep_us(10);  // Very short glitch
        gpio_put(CLOCK_CONTROL_PIN, 1);
        sleep_us(10);
    }
}

int test_fault_recovery(fault_type_t fault_type) {
    printf("Testing recovery from fault type %d\n", fault_type);

    // Record system state before fault
    bool system_ready_before = check_system_ready();

    // Inject fault
    switch (fault_type) {
        case FAULT_POWER_GLITCH:
            inject_power_glitch(50);
            break;
        case FAULT_CLOCK_GLITCH:
            inject_clock_glitch(10);
            break;
        // Add more fault types...
    }

    // Check recovery
    sleep_ms(1000);  // Allow recovery time
    bool system_ready_after = check_system_ready();

    if (system_ready_after) {
        printf("System recovered successfully\n");
        return 1;
    } else {
        printf("System failed to recover\n");
        return 0;
    }
}
\end{lstlisting}
\end{frame}

\begin{frame}
\frametitle{Real-Time Debugging Strategies}
\textbf{Challenges in real-time systems:}
\begin{itemize}
    \item Timing-sensitive operations
    \item Interrupt-driven behavior
    \item Race conditions
    \item Resource contention
    \item Non-deterministic execution
\end{itemize}

\vspace{0.5cm}
\textbf{Debugging techniques:}
\begin{itemize}
    \item \textbf{Non-intrusive logging:} Minimal impact on timing
    \item \textbf{State machines:} Track system states
    \item \textbf{Watchdog timers:} Detect system hangs
    \item \textbf{Performance counters:} Monitor resource usage
    \item \textbf{Logic analyzers:} Capture signal timing
\end{itemize}

\vspace{0.5cm}
\textbf{Best practices:}
\begin{itemize}
    \item Keep interrupt handlers short
    \item Use atomic operations for shared data
    \item Implement proper synchronization
    \item Test under worst-case conditions
\end{itemize}
\end{frame}

\begin{frame}[fragile]
\frametitle{Performance Optimization Example}
\begin{lstlisting}[language=C]
#include "pico/stdlib.h"
#include "hardware/timer.h"

// Unoptimized version
int validate_registers_slow(uint32_t *registers, int count) {
    int errors = 0;
    for (int i = 0; i < count; i++) {
        uint32_t expected = calculate_expected_value(i);  // Expensive!
        if (registers[i] != expected) {
            errors++;
            printf("Register %d: expected 0x%08lX, got 0x%08lX\n",
                   i, expected, registers[i]);
        }
    }
    return errors;
}

// Optimized version
int validate_registers_fast(uint32_t *registers, int count) {
    static uint32_t expected_values[MAX_REGISTERS];  // Pre-calculated
    static bool values_initialized = false;

    // Initialize expected values once
    if (!values_initialized) {
        for (int i = 0; i < count; i++) {
            expected_values[i] = calculate_expected_value(i);
        }
        values_initialized = true;
    }

    int errors = 0;
    for (int i = 0; i < count; i++) {
        if (registers[i] != expected_values[i]) {
            errors++;
            // Minimal logging for speed
            error_log[errors-1] = i;  // Just store index
        }
    }
    return errors;
}

void benchmark_validation_functions(void) {
    uint32_t test_registers[100];
    // Initialize test data...

    uint64_t slow_time = measure_execution_time_us(
        validate_registers_slow_wrapper);
    uint64_t fast_time = measure_execution_time_us(
        validate_registers_fast_wrapper);

    printf("Slow version: %llu us\n", slow_time);
    printf("Fast version: %llu us\n", fast_time);
    printf("Speedup: %%.1fx\n", (float)slow_time / fast_time);
}
\end{lstlisting}
\end{frame}

\begin{frame}
\frametitle{Interactive Poll: Debugging Challenge}
\begin{center}
\Large Your validation system occasionally hangs. What's your first debugging step?
\end{center}

\textbf{A)} Add more printf statements everywhere

\textbf{B)} Implement a watchdog timer

\textbf{C)} Use a logic analyzer to check signals

\textbf{D)} Rewrite the entire system

\pause

\begin{alertblock}{Answer}
\textbf{B)} Watchdog timer - Detect and recover from hangs automatically!
\end{alertblock}

\vspace{0.5cm}
\textbf{Then} use logic analyzer and targeted debugging to find root cause.
\end{frame}

\begin{frame}[fragile]
\frametitle{Watchdog Implementation}
\begin{lstlisting}[language=C]
#include "pico/stdlib.h"
#include "hardware/watchdog.h"

#define WATCHDOG_TIMEOUT_MS 5000  // 5 second timeout

void setup_watchdog_protection(void) {
    // Enable watchdog with 5 second timeout
    watchdog_enable(WATCHDOG_TIMEOUT_MS, 1);
    printf("Watchdog enabled with %d ms timeout\n", WATCHDOG_TIMEOUT_MS);
}

void validation_main_loop(void) {
    setup_watchdog_protection();

    while (true) {
        // Reset watchdog at start of each cycle
        watchdog_update();

        printf("Starting validation cycle...\n");

        // Run validation tests
        int gpio_result = run_gpio_tests();
        watchdog_update();  // Reset after each major operation

        int adc_result = run_adc_tests();
        watchdog_update();

        int timing_result = run_timing_tests();
        watchdog_update();

        // Report results
        printf("Cycle complete: GPIO=%d, ADC=%d, Timing=%d\n",
               gpio_result, adc_result, timing_result);

        sleep_ms(1000);
        // Watchdog will be reset at start of next cycle
    }
}

// If system hangs, watchdog will reset the system
// Check reset reason on startup:
void check_reset_reason(void) {
    if (watchdog_caused_reboot()) {
        printf("WARNING: System was reset by watchdog!\n");
        printf("Possible system hang detected.\n");
    }
}
\end{lstlisting}
\end{frame}

\begin{frame}
\frametitle{Lab Preview: Complete Hardware Validator}
\textbf{This afternoon you'll build:}
\begin{itemize}
    \item Comprehensive MicroBlaze-V peripheral validation suite
    \item Advanced debugging with printf and LED indicators
    \item Hardware-in-the-loop testing framework
    \item Fault injection and recovery testing
    \item Performance-optimized validation routines
    \item Real-time constraint validation
\end{itemize}

\vspace{0.5cm}
\textbf{Advanced features:}
\begin{itemize}
    \item Interrupt-based event handling
    \item Watchdog protection system
    \item Automated test reporting
    \item Statistical analysis of results
    \item AI-assisted optimization
\end{itemize}
\end{frame}

\begin{frame}
\frametitle{Hardware Safety Considerations}
\textbf{Protecting your hardware:}
\begin{itemize}
    \item \textbf{Voltage limits:} Never exceed 3.3V on MicroBlaze-V pins
    \item \textbf{Current limits:} Maximum 12mA per GPIO pin
    \item \textbf{ESD protection:} Use anti-static precautions
    \item \textbf{Power sequencing:} Proper startup/shutdown procedures
    \item \textbf{Thermal management:} Monitor temperature during stress tests
\end{itemize}

\vspace{0.5cm}
\textbf{Validation safety practices:}
\begin{itemize}
    \item Start with low-stress tests
    \item Implement emergency shutdown procedures
    \item Use current-limited power supplies
    \item Monitor system health continuously
    \item Have backup hardware available
\end{itemize}
\end{frame}

\begin{frame}
\frametitle{Key Takeaways - Day 5}
\begin{itemize}
    \item \textbf{Hardware debugging is different:} Use printf, LEDs, and external tools
    \item \textbf{Peripherals enable validation:} GPIO, ADC, timers are your tools
    \item \textbf{Real-time matters:} Timing constraints are critical
    \item \textbf{Fault injection tests robustness:} Controlled failures reveal weaknesses
    \item \textbf{Optimization is essential:} Embedded systems demand efficiency
    \item \textbf{Safety first:} Protect hardware during aggressive testing
\end{itemize}

\vspace{0.5cm}
\begin{center}
\textbf{You're now a hardware validation expert!}
\end{center}
\end{frame}

\begin{frame}
\frametitle{Tonight's Homework}
\textbf{Enhance your hardware validator:}
\begin{enumerate}
    \item Add comprehensive peripheral testing
    \item Implement fault injection scenarios
    \item Optimize critical validation routines
    \item Add watchdog protection and error recovery
    \item Create detailed performance benchmarks
    \item Use AI to suggest hardware-specific optimizations
\end{enumerate}

\vspace{0.5cm}
\textbf{Advanced challenges:}
\begin{itemize}
    \item Implement interrupt-based testing
    \item Add statistical analysis of test results
    \item Create automated test report generation
    \item Explore advanced MicroBlaze-V features (PIO, DMA)
\end{itemize}
\end{frame}

\begin{frame}
\frametitle{Tomorrow Preview: Capstone Project}
\begin{center}
\Large The Grand Finale!
\end{center}

\textbf{Day 6 - Capstone Project:}
\begin{itemize}
    \item Integrate all learned concepts
    \item Design and implement complete validation system
    \item Team collaboration and project management
    \item Professional presentation of results
    \item Portfolio development and career planning
    \item Course wrap-up and next steps
\end{itemize}

\vspace{0.5cm}
\begin{alertblock}{Showcase Day}
Tomorrow you'll demonstrate your validation engineering skills!
\end{alertblock}
\end{frame}

\begin{frame}
\frametitle{Questions \& Discussion}
\begin{center}
\Large Ready to validate real hardware?
\end{center}

\begin{itemize}
    \item Hardware debugging techniques questions?
    \item MicroBlaze-V peripheral programming challenges?
    \item Real-time optimization strategies?
    \item Fault injection implementation ideas?
    \item Excited about tomorrow's capstone project?
\end{itemize}

\vspace{1cm}
\begin{center}
\textbf{Time to build your hardware validation masterpiece!}
\end{center}
\end{frame}

\end{document}

