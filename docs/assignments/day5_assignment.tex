\documentclass[11pt,a4paper]{article}
\usepackage[utf8]{inputenc}
\usepackage[T1]{fontenc}
\usepackage{geometry}
\usepackage{graphicx}
\usepackage{xcolor}
\usepackage{listings}
\usepackage{fancyhdr}
\usepackage{titlesec}
\usepackage{hyperref}
\usepackage{enumitem}
\usepackage{booktabs}
\usepackage{array}

% Page setup
\geometry{margin=1in}
\pagestyle{fancy}
\fancyhf{}
\fancyhead[L]{Day 5 Assignment: Hardware Debugging and RP2040 Peripherals Lab}
\fancyhead[R]{\thepage}
\fancyfoot[C]{C Programming for Post-Silicon Validation Engineers}

% Colors
\definecolor{codeblue}{RGB}{0,102,204}
\definecolor{codegray}{RGB}{128,128,128}
\definecolor{codegreen}{RGB}{0,128,0}
\definecolor{backcolour}{RGB}{245,245,245}

% Code listing style
\lstdefinestyle{cstyle}{
    backgroundcolor=\color{backcolour},
    commentstyle=\color{codegreen},
    keywordstyle=\color{codeblue},
    numberstyle=\tiny\color{codegray},
    stringstyle=\color{red},
    basicstyle=\ttfamily\footnotesize,
    breakatwhitespace=false,
    breaklines=true,
    captionpos=b,
    keepspaces=true,
    numbers=left,
    numbersep=5pt,
    showspaces=false,
    showstringspaces=false,
    showtabs=false,
    tabsize=2,
    frame=single
}

\lstset{style=cstyle}

% Title formatting
\titleformat{\section}{\Large\bfseries\color{codeblue}}{\thesection}{1em}{}
\titleformat{\subsection}{\large\bfseries}{\thesubsection}{1em}{}

\hypersetup{
    colorlinks=true,
    linkcolor=codeblue,
    filecolor=magenta,
    urlcolor=cyan,
    pdftitle={Day 5 Assignment - Hardware Debugging and RP2040 Peripherals Lab},
    pdfauthor={Yahwista Salomo},
}

\begin{document}

% Header
\begin{center}
    {\Huge\bfseries\color{codeblue} DAY 5 ASSIGNMENT}\\[0.5cm]
    {\Large Hardware Debugging and RP2040 Peripherals Lab}\\[0.3cm]
    {\large Complete Hardware-in-the-Loop Validation System}\\[0.2cm]
    {\normalsize Due: End of Day 5 + Homework Extension}
\end{center}

\vspace{1cm}

\section{Assignment Overview}

\subsection{Learning Objectives}
By completing this assignment, you will:
\begin{itemize}
    \item Master advanced debugging techniques for embedded systems
    \item Program RP2040 peripherals (GPIO, ADC, timers, interrupts)
    \item Implement hardware-in-the-loop testing frameworks
    \item Create fault injection and recovery systems
    \item Optimize code for real-time embedded constraints
    \item Build production-ready validation systems
\end{itemize}

\subsection{Assignment Context}
You will create a comprehensive hardware validation system that runs on the RP2040, utilizing real peripherals to perform actual hardware testing. This represents the culmination of your embedded programming skills and demonstrates real-world validation engineering capabilities.

\section{Part 1: In-Class Lab (3.5 hours)}

\subsection{Setup and Hardware Preparation}
\begin{enumerate}
    \item Accept the Day 5 GitHub Classroom assignment
    \item Verify RP2040 hardware is functional and connected
    \item Test serial communication with your development environment
    \item Ensure all previous day's code is committed and builds successfully
    \item Create new branch: \texttt{git checkout -b day5-hardware-validation}
\end{enumerate}

\subsection{Task 1: Advanced Debugging Infrastructure (60 minutes)}

\textbf{Requirements:}
\begin{itemize}
    \item Implement multi-level debug output system
    \item Create LED-based visual debugging indicators
    \item Add performance monitoring and timing analysis
    \item Implement watchdog protection system
\end{itemize}

\textbf{Debug Infrastructure Files:}
\begin{itemize}
    \item \texttt{include/debug\_system.h}
    \item \texttt{src/debug\_system.c}
    \item \texttt{include/performance\_monitor.h}
    \item \texttt{src/performance\_monitor.c}
\end{itemize}

\textbf{Debug System Interface:}
\begin{lstlisting}[language=C]
// debug_system.h
#ifndef DEBUG_SYSTEM_H
#define DEBUG_SYSTEM_H

#include <stdint.h>
#include <stdbool.h>

// Debug levels
typedef enum {
    DEBUG_LEVEL_NONE = 0,
    DEBUG_LEVEL_ERROR,
    DEBUG_LEVEL_WARNING,
    DEBUG_LEVEL_INFO,
    DEBUG_LEVEL_VERBOSE
} debug_level_t;

// LED debug patterns
typedef enum {
    LED_PATTERN_OFF,
    LED_PATTERN_SOLID,
    LED_PATTERN_SLOW_BLINK,
    LED_PATTERN_FAST_BLINK,
    LED_PATTERN_HEARTBEAT,
    LED_PATTERN_ERROR_CODE
} led_pattern_t;

// Debug system functions
void debug_init(debug_level_t level);
void debug_print(debug_level_t level, const char* format, ...);
void debug_led_set_pattern(led_pattern_t pattern);
void debug_led_show_error_code(uint8_t error_code);
void debug_assert(bool condition, const char* message);

// Watchdog functions
void watchdog_init(uint32_t timeout_ms);
void watchdog_feed(void);
bool watchdog_caused_reset(void);

// Performance monitoring
void perf_counter_start(const char* name);
void perf_counter_end(const char* name);
void perf_print_report(void);
void perf_reset_counters(void);

#endif // DEBUG_SYSTEM_H
\end{lstlisting}

\subsection{Task 2: GPIO Comprehensive Testing Suite (60 minutes)}

\textbf{Requirements:}
\begin{itemize}
    \item Test all available GPIO pins systematically
    \item Implement input/output functionality validation
    \item Add pull-up/pull-down resistance testing
    \item Create GPIO stress testing and reliability checks
\end{itemize}

\textbf{GPIO Testing Files:}
\begin{itemize}
    \item \texttt{include/gpio\_validator.h}
    \item \texttt{src/gpio\_validator.c}
\end{itemize}

\textbf{GPIO Test Framework:}
\begin{lstlisting}[language=C]
// gpio_validator.h
#ifndef GPIO_VALIDATOR_H
#define GPIO_VALIDATOR_H

#include <stdint.h>
#include <stdbool.h>

#define MAX_GPIO_PINS 29
#define GPIO_TEST_PATTERNS 8

typedef struct {
    uint8_t pin;
    bool output_test_pass;
    bool input_test_pass;
    bool pullup_test_pass;
    bool pulldown_test_pass;
    bool speed_test_pass;
    uint32_t toggle_frequency_hz;
    char error_message[64];
} gpio_test_result_t;

typedef struct {
    gpio_test_result_t results[MAX_GPIO_PINS];
    int pins_tested;
    int pins_passed;
    int pins_failed;
    float overall_success_rate;
} gpio_test_summary_t;

// GPIO testing functions
int gpio_validator_init(void);
int test_single_gpio_pin(uint8_t pin, gpio_test_result_t* result);
int test_gpio_range(uint8_t start_pin, uint8_t end_pin, gpio_test_summary_t* summary);
int test_gpio_patterns(uint8_t pin, uint32_t* patterns, int pattern_count);
int test_gpio_speed(uint8_t pin, uint32_t target_frequency_hz);
void print_gpio_test_report(const gpio_test_summary_t* summary);

// Advanced GPIO tests
int test_gpio_interrupt_functionality(uint8_t pin);
int test_gpio_drive_strength(uint8_t pin);
int stress_test_gpio_reliability(uint8_t pin, uint32_t iterations);

#endif // GPIO_VALIDATOR_H
\end{lstlisting}

\subsection{Task 3: ADC and Analog Validation (60 minutes)}

\textbf{Requirements:}
\begin{itemize}
    \item Implement comprehensive ADC testing across all channels
    \item Add voltage reference validation and calibration
    \item Create noise analysis and signal quality assessment
    \item Implement temperature sensor validation
\end{itemize}

\textbf{ADC Testing Files:}
\begin{itemize}
    \item \texttt{include/adc\_validator.h}
    \item \texttt{src/adc\_validator.c}
\end{itemize}

\textbf{ADC Validation Framework:}
\begin{lstlisting}[language=C]
// adc_validator.h
#ifndef ADC_VALIDATOR_H
#define ADC_VALIDATOR_H

#include <stdint.h>
#include <stdbool.h>

#define ADC_CHANNELS 4
#define ADC_SAMPLES_PER_TEST 1000
#define ADC_REFERENCE_VOLTAGE 3.3f

typedef struct {
    uint8_t channel;
    float voltage_reading;
    float expected_voltage;
    float voltage_error;
    float noise_level_mv;
    uint16_t raw_value;
    bool calibration_valid;
    bool within_tolerance;
} adc_test_result_t;

typedef struct {
    adc_test_result_t channel_results[ADC_CHANNELS];
    float reference_voltage_measured;
    float temperature_celsius;
    bool temperature_sensor_valid;
    int channels_tested;
    int channels_passed;
} adc_test_summary_t;

// ADC validation functions
int adc_validator_init(void);
int test_adc_channel(uint8_t channel, float expected_voltage, adc_test_result_t* result);
int test_all_adc_channels(adc_test_summary_t* summary);
float measure_adc_noise(uint8_t channel, int sample_count);
int calibrate_adc_channel(uint8_t channel);
float read_internal_temperature(void);

// Advanced ADC tests
int test_adc_linearity(uint8_t channel, float* test_voltages, int voltage_count);
int test_adc_settling_time(uint8_t channel);
int stress_test_adc_stability(uint8_t channel, uint32_t duration_ms);

#endif // ADC_VALIDATOR_H
\end{lstlisting>

\subsection{Task 4: Timer and Interrupt Validation (60 minutes)}

\textbf{Requirements:}
\begin{itemize}
    \item Implement precise timing measurement and validation
    \item Create interrupt-based event detection system
    \item Add timer accuracy and drift analysis
    \item Implement real-time constraint validation
\end{itemize}

\textbf{Timer Testing Files:}
\begin{itemize}
    \item \texttt{include/timer\_validator.h}
    \item \texttt{src/timer\_validator.c}
\end{itemize}

\textbf{Timer Validation Framework:}
\begin{lstlisting}[language=C]
// timer_validator.h
#ifndef TIMER_VALIDATOR_H
#define TIMER_VALIDATOR_H

#include <stdint.h>
#include <stdbool.h>

typedef struct {
    uint32_t expected_interval_us;
    uint32_t measured_interval_us;
    int32_t timing_error_us;
    float timing_accuracy_percent;
    bool within_tolerance;
} timer_test_result_t;

typedef struct {
    volatile uint32_t interrupt_count;
    volatile uint64_t last_interrupt_time;
    volatile bool test_complete;
    uint32_t expected_interrupts;
    uint32_t test_duration_ms;
} interrupt_test_context_t;

// Timer validation functions
int timer_validator_init(void);
int test_timer_accuracy(uint32_t interval_us, uint32_t duration_ms, timer_test_result_t* result);
int test_interrupt_timing(uint32_t interval_us, uint32_t count, interrupt_test_context_t* context);
uint64_t measure_function_execution_time(void (*function)(void));
int validate_real_time_constraints(void (*function)(void), uint32_t max_time_us);

// Advanced timer tests
int test_timer_drift_over_time(uint32_t interval_us, uint32_t duration_minutes);
int test_interrupt_latency(void);
int stress_test_timer_under_load(void);

#endif // TIMER_VALIDATOR_H
\end{lstlisting>

\subsection{Task 5: Fault Injection and Recovery Testing (30 minutes)}

\textbf{Requirements:}
\begin{itemize}
    \item Implement controlled fault injection mechanisms
    \item Create system recovery and resilience testing
    \item Add error detection and classification
    \item Implement graceful degradation strategies
\end{itemize}

\textbf{Fault Injection Framework:}
\begin{lstlisting}[language=C]
// fault_injection.h
#ifndef FAULT_INJECTION_H
#define FAULT_INJECTION_H

#include <stdint.h>
#include <stdbool.h>

typedef enum {
    FAULT_TYPE_NONE,
    FAULT_TYPE_POWER_GLITCH,
    FAULT_TYPE_CLOCK_GLITCH,
    FAULT_TYPE_MEMORY_CORRUPTION,
    FAULT_TYPE_INTERRUPT_STORM,
    FAULT_TYPE_PERIPHERAL_FAILURE
} fault_type_t;

typedef struct {
    fault_type_t type;
    uint32_t duration_ms;
    uint32_t intensity;
    bool recovery_successful;
    uint32_t recovery_time_ms;
    char description[128];
} fault_test_result_t;

// Fault injection functions
int fault_injection_init(void);
int inject_fault(fault_type_t type, uint32_t duration_ms, uint32_t intensity);
int test_system_recovery(fault_type_t fault_type, fault_test_result_t* result);
bool check_system_health(void);
int implement_graceful_degradation(fault_type_t active_fault);

#endif // FAULT_INJECTION_H
\end{lstlisting>

\section{Part 2: Homework Extension (2 hours)}

\subsection{Task 6: Advanced Peripheral Integration}

\textbf{Requirements:}
\begin{itemize}
    \item Integrate UART, SPI, and I2C communication testing
    \item Add PWM signal generation and validation
    \item Implement DMA-based data transfer validation
    \item Create multi-peripheral coordination tests
\end{itemize}

\textbf{Communication Protocol Testing:}
\begin{lstlisting}[language=C]
// communication_validator.h
typedef struct {
    bool uart_loopback_test;
    bool spi_transfer_test;
    bool i2c_device_scan;
    uint32_t uart_baud_rate_actual;
    uint32_t spi_frequency_actual;
    bool communication_errors_detected;
} communication_test_results_t;

int test_uart_communication(uint32_t baud_rate, communication_test_results_t* results);
int test_spi_communication(uint32_t frequency, communication_test_results_t* results);
int test_i2c_communication(communication_test_results_t* results);
int test_pwm_output(uint8_t pin, uint32_t frequency, float duty_cycle);
\end{lstlisting}

\subsection{Task 7: System Integration and Stress Testing}

\textbf{Requirements:}
\begin{itemize}
    \item Create comprehensive system-level validation suite
    \item Implement concurrent peripheral testing
    \item Add thermal and power consumption monitoring
    \item Create long-duration reliability testing
\end{itemize}

\textbf{System Integration Framework:}
\begin{lstlisting}[language=C]
// system_validator.h
typedef struct {
    float cpu_utilization_percent;
    uint32_t memory_usage_bytes;
    float temperature_celsius;
    float power_consumption_mw;
    uint32_t uptime_seconds;
    uint32_t error_count;
    bool system_stable;
} system_health_t;

int run_comprehensive_system_test(uint32_t duration_minutes);
int monitor_system_health(system_health_t* health);
int stress_test_all_peripherals(uint32_t duration_minutes);
int validate_system_under_load(void);
\end{lstlisting>

\subsection{Task 8: Production-Ready Features}

\textbf{Requirements:}
\begin{itemize}
    \item Add configuration management and persistence
    \item Implement test result logging and export
    \item Create automated test scheduling
    \item Add remote monitoring and control capabilities
\end{itemize}

\section{Submission Requirements}

\subsection{Hardware Verification Checklist}
\begin{itemize}
    \item[$\square$] All GPIO pins tested and documented
    \item[$\square$] ADC channels calibrated and validated
    \item[$\square$] Timer accuracy verified within specifications
    \item[$\square$] Interrupt system functioning correctly
    \item[$\square$] Fault injection and recovery working
    \item[$\square$] System runs stably for extended periods
    \item[$\square$] Debug output clear and informative
    \item[$\square$] LED indicators provide useful feedback
\end{itemize}

\subsection{Documentation Requirements}
\begin{itemize}
    \item \textbf{HARDWARE\_TEST\_REPORT.md:} Comprehensive test results and analysis
    \item \textbf{DEBUGGING\_GUIDE.md:} Guide to using the debug infrastructure
    \item \textbf{PERIPHERAL\_REFERENCE.md:} Documentation of all peripheral tests
    \item \textbf{PERFORMANCE\_ANALYSIS.md:} Timing and performance measurements
    \item \textbf{FAULT\_TESTING\_RESULTS.md:} Fault injection test outcomes
\end{itemize}

\subsection{Code Quality Standards}
\begin{itemize}
    \item \textbf{Real-Time Safety:} No blocking operations in interrupt handlers
    \item \textbf{Resource Management:} Proper initialization and cleanup
    \item \textbf{Error Handling:} Comprehensive error detection and recovery
    \item \textbf{Performance:} Optimized for embedded constraints
    \item \textbf{Reliability:} Robust operation under stress conditions
\end{itemize}

\section{Grading Rubric}

\begin{center}
\begin{tabular}{|l|c|l|}
\hline
\textbf{Component} & \textbf{Points} & \textbf{Criteria} \\
\hline
Task 1: Debug Infrastructure & 20 & Comprehensive debugging tools \\
Task 2: GPIO Testing & 20 & Complete GPIO validation suite \\
Task 3: ADC Validation & 20 & Accurate analog measurements \\
Task 4: Timer/Interrupt & 20 & Precise timing validation \\
Task 5: Fault Injection & 15 & Robust fault testing \\
Task 6: Advanced Peripherals & 15 & Communication protocol testing \\
Task 7: System Integration & 15 & Comprehensive system tests \\
Task 8: Production Features & 10 & Professional-grade features \\
Hardware Verification & 20 & Actual hardware testing \\
Documentation & 15 & Complete technical documentation \\
\hline
\textbf{Total} & \textbf{170} & \textbf{Production-ready validation system} \\
\hline
\end{tabular}
\end{center>

\section{Hardware Safety and Best Practices}

\subsection{Safety Guidelines}
\begin{itemize}
    \item \textbf{Voltage Limits:} Never exceed 3.3V on any RP2040 pin
    \item \textbf{Current Limits:} Maximum 12mA per GPIO pin, 50mA total
    \item \textbf{ESD Protection:} Use anti-static precautions when handling hardware
    \item \textbf{Power Management:} Monitor power consumption during stress tests
    \item \textbf{Thermal Management:} Monitor temperature during extended testing
\end{itemize}

\subsection{Testing Best Practices}
\begin{itemize}
    \item Start with low-stress tests and gradually increase intensity
    \item Always implement emergency shutdown procedures
    \item Use current-limited power supplies for safety
    \item Monitor system health continuously during tests
    \item Keep backup hardware available for critical testing
\end{itemize}

\section{Troubleshooting Guide}

\subsection{Common Hardware Issues}
\begin{itemize}
    \item \textbf{GPIO not responding:} Check pin initialization and direction
    \item \textbf{ADC readings unstable:} Verify reference voltage and grounding
    \item \textbf{Timer inaccuracy:} Check clock configuration and crystal accuracy
    \item \textbf{Interrupt not firing:} Verify interrupt enable and handler registration
    \item \textbf{System crashes:} Check stack usage and memory allocation
\end{itemize}

\subsection{Debug Strategies}
\begin{itemize}
    \item Use LED patterns to indicate system state without serial output
    \item Implement progressive testing (test one peripheral at a time)
    \item Use oscilloscope or logic analyzer for signal verification
    \item Monitor power consumption for abnormal behavior detection
    \item Implement comprehensive logging for post-mortem analysis
\end{itemize>

\section{Performance Optimization}

\subsection{Embedded Optimization Techniques}
\begin{itemize}
    \item \textbf{Memory Usage:} Minimize dynamic allocation, use static buffers
    \item \textbf{CPU Efficiency:} Optimize critical loops, use appropriate data types
    \item \textbf{Interrupt Latency:} Keep interrupt handlers short and fast
    \item \textbf{Power Consumption:} Use sleep modes when appropriate
    \item \textbf{Code Size:} Optimize for flash memory constraints
\end{itemize}

\subsection{Real-Time Considerations}
\begin{itemize}
    \item Ensure deterministic execution times for critical functions
    \item Avoid blocking operations in time-critical code paths
    \item Use appropriate priority levels for different tasks
    \item Implement proper synchronization for shared resources
    \item Monitor worst-case execution times under all conditions
\end{itemize>

\section{Extension Opportunities}

\subsection{Advanced Features}
\begin{itemize}
    \item \textbf{Multi-Core Utilization:} Use both RP2040 cores for parallel testing
    \item \textbf{Custom Protocols:} Implement proprietary communication protocols
    \item \textbf{Machine Learning:} Add anomaly detection using TensorFlow Lite
    \item \textbf{Wireless Communication:} Add WiFi or Bluetooth validation
    \item \textbf{External Sensors:} Interface with I2C/SPI sensors for environmental monitoring
\end{itemize}

\subsection{Professional Development}
\begin{itemize}
    \item Study commercial ATE (Automated Test Equipment) systems
    \item Research industry standards for hardware validation
    \item Explore professional debugging tools and techniques
    \item Investigate certification requirements for embedded systems
    \item Learn about production testing and quality assurance processes
\end{itemize>

\section{Success Tips}

\begin{itemize}
    \item \textbf{Hardware First:} Verify basic hardware functionality before complex tests
    \item \textbf{Incremental Development:} Build and test one feature at a time
    \item \textbf{Document Everything:} Record all test results and observations
    \item \textbf{Safety First:} Always consider hardware safety in your designs
    \item \textbf{Think Like a Validator:} Consider what could go wrong and test for it
    \item \textbf{Performance Matters:} Optimize for the embedded environment
\end{itemize}

\vspace{1cm}

\begin{center}
\textbf{You're now a hardware validation expert!}\\
\textit{These skills directly translate to professional validation engineering roles.}
\end{center>

\end{document>

