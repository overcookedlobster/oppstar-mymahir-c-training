\documentclass[11pt,a4paper]{article}
\usepackage[utf8]{inputenc}
\usepackage[T1]{fontenc}
\usepackage{geometry}
\usepackage{graphicx}
\usepackage{xcolor}
\usepackage{listings}
\usepackage{fancyhdr}
\usepackage{titlesec}
\usepackage{hyperref}
\usepackage{enumitem}
\usepackage{booktabs}
\usepackage{array}

% Page setup
\geometry{margin=1in}
\setlength{\headheight}{13.59999pt}
\pagestyle{fancy}
\fancyhf{}
\fancyhead[L]{Day 2 Assignment: Control Flow and Debugging Lab}
\fancyhead[R]{\thepage}
\fancyfoot[C]{C Programming for Post-Silicon Validation Engineers}

% Colors
\definecolor{codeblue}{RGB}{0,102,204}
\definecolor{codegray}{RGB}{128,128,128}
\definecolor{codegreen}{RGB}{0,128,0}
\definecolor{backcolour}{RGB}{245,245,245}

% Code listing style
\lstdefinestyle{cstyle}{
    backgroundcolor=\color{backcolour},
    commentstyle=\color{codegreen},
    keywordstyle=\color{codeblue},
    numberstyle=\tiny\color{codegray},
    stringstyle=\color{red},
    basicstyle=\ttfamily\footnotesize,
    breakatwhitespace=false,
    breaklines=true,
    captionpos=b,
    keepspaces=true,
    numbers=left,
    numbersep=5pt,
    showspaces=false,
    showstringspaces=false,
    showtabs=false,
    tabsize=2,
    frame=single
}

\lstset{style=cstyle}

% Title formatting
\titleformat{\section}{\Large\bfseries\color{codeblue}}{\thesection}{1em}{}
\titleformat{\subsection}{\large\bfseries}{\thesubsection}{1em}{}

\hypersetup{
    colorlinks=true,
    linkcolor=codeblue,
    filecolor=magenta,
    urlcolor=cyan,
    pdftitle={Day 2 Assignment - Control Flow and Debugging Lab},
    pdfauthor={Yahwista Salomo},
}

\begin{document}

% Header
\begin{center}
    {\Huge\bfseries\color{codeblue} DAY 2 ASSIGNMENT}\\[0.5cm]
    {\Large Control Flow and Debugging Lab}\\[0.3cm]
    {\large Register Monitoring and Test Automation}\\[0.2cm]
    {\normalsize Due: End of Day 2 + Homework Extension}
\end{center}

\vspace{1cm}

\section{Assignment Overview}

\subsection{Learning Objectives}
By completing this assignment, you will:
\begin{itemize}
    \item Learn to create and call simple functions in C
    \item Practice basic loops (for and while) for repetitive tasks
    \item Understand function parameters and return values
    \item Get introduced to basic arrays and array processing
    \item Learn basic debugging techniques using printf
    \item Build simple programs that use functions and loops together
\end{itemize}

\subsection{Prerequisites}
Before starting this assignment, you should have completed Day 1 and be comfortable with:
\begin{itemize}
    \item Basic C syntax and variable declarations
    \item Using printf and scanf for input/output
    \item Simple if-else statements and comparisons
    \item Compiling C programs with GCC
    \item Basic Git operations (covered in previous sessions)
\end{itemize}

\subsection{Assignment Context}
You will learn fundamental programming concepts by building simple validation tools. We'll start with basic functions, then add loops to automate repetitive tasks, and finally combine everything into useful programs. The focus is on mastering C fundamentals through practical, step-by-step exercises.

\section{Part 1: In-Class Lab (3 hours)}

\subsection{Setup and Repository}
\begin{enumerate}
    \item Accept the Day 2 GitHub Classroom assignment
    \item Clone your repository and navigate to the project directory
    \item Verify your Day 1 work is committed and pushed
    \item Create a new branch: \texttt{git checkout -b day2-development}
\end{enumerate}

\subsection{Task 1: Your First Functions (45 minutes)}

\textbf{File:} \texttt{simple\_functions.c}

\textbf{Requirements:}
\begin{itemize}
    \item Create simple functions that perform basic calculations
    \item Learn to call functions from main()
    \item Practice functions with no parameters and simple return values
    \item Understand function scope and local variables
\end{itemize}

\textbf{Starter Code Template:}
\begin{lstlisting}[language=C]
#include <stdio.h>

// TODO: Create a function that prints a welcome message
void print_welcome() {
    // TODO: Print "Welcome to Day 2 - Functions and Loops!"
}

// TODO: Create a function that returns the maximum voltage limit
float get_max_voltage() {
    // TODO: Return 3.6 as the maximum voltage
}

// TODO: Create a function that calculates power (P = V * I)
float calculate_power(float voltage, float current) {
    // TODO: Calculate and return power
}

int main() {
    // TODO: Call print_welcome function

    // TODO: Get max voltage and display it

    // TODO: Test calculate_power with sample values
    float test_voltage = 3.3;
    float test_current = 0.5;
    // TODO: Call calculate_power and display result

    return 0;
}
\end{lstlisting}

\textbf{Expected Output Example:}
\begin{verbatim}
Welcome to Day 2 - Functions and Loops!
Maximum voltage limit: 3.60V
Testing power calculation:
Voltage: 3.30V, Current: 0.50A
Power consumption: 1.65W
\end{verbatim}

\textbf{Step-by-Step Guide:}
\begin{enumerate}
    \item Complete the \texttt{print\_welcome()} function first
    \item Add the \texttt{get\_max\_voltage()} function - just return 3.6
    \item Implement \texttt{calculate\_power()} - multiply voltage by current
    \item In main(), call each function and display the results
    \item Test with different voltage and current values
\end{enumerate}

\subsection{Task 2: Simple Loop Practice (45 minutes)}

\textbf{File:} \texttt{loop\_practice.c}

\textbf{Requirements:}
\begin{itemize}
    \item Practice basic for loops with counting
    \item Learn simple while loops for repetitive tasks
    \item Use loops to process multiple values
    \item Understand loop control and termination
\end{itemize}

\textbf{Loop Exercises:}
\begin{enumerate}
    \item \textbf{Counting Loop:} Print numbers 1 to 10
    \item \textbf{Voltage Testing:} Test 5 different voltage values in a loop
    \item \textbf{Sum Calculator:} Add up numbers from 1 to N
    \item \textbf{Pattern Printing:} Print simple patterns using loops
\end{enumerate}

\textbf{Starter Code Template:}
\begin{lstlisting}[language=C]
#include <stdio.h>

int main() {
    // TODO: Exercise 1 - Count from 1 to 10
    printf("=== Counting Exercise ===\n");
    for (int i = 1; i <= 10; i++) {
        // TODO: Print the number
    }

    // TODO: Exercise 2 - Test multiple voltages
    printf("\n=== Voltage Testing ===\n");
    float voltages[] = {1.8, 2.5, 3.3, 3.6, 4.0};
    // TODO: Use a loop to test each voltage

    // TODO: Exercise 3 - Calculate sum
    printf("\n=== Sum Calculator ===\n");
    int sum = 0;
    // TODO: Add numbers 1 to 5 using a loop

    return 0;
}
\end{lstlisting}

\subsection{Task 3: Basic Arrays and Functions (60 minutes)}

\textbf{File:} \texttt{array\_functions.c}

\textbf{Requirements:}
\begin{itemize}
    \item Learn to declare and initialize arrays
    \item Create functions that work with arrays
    \item Use loops to process array elements
    \item Practice passing arrays to functions
\end{itemize}

\textbf{Array Exercises:}
\begin{enumerate}
    \item \textbf{Array Declaration:} Create arrays of test values
    \item \textbf{Array Processing:} Use loops to check each element
    \item \textbf{Function with Arrays:} Pass arrays to validation functions
    \item \textbf{Array Statistics:} Calculate average, min, max values
\end{enumerate}

\textbf{Starter Code Template:}
\begin{lstlisting}[language=C]
#include <stdio.h>

// TODO: Create a function to check if voltage is valid
int is_voltage_valid(float voltage) {
    // TODO: Return 1 if voltage is between 1.8 and 3.6, 0 otherwise
}

// TODO: Create a function to find average of array
float calculate_average(float values[], int size) {
    // TODO: Calculate and return average
}

int main() {
    // TODO: Declare array of test voltages
    float test_voltages[] = {1.8, 2.5, 3.3, 3.6, 4.0};
    int array_size = 5;

    // TODO: Test each voltage using a loop and function
    printf("=== Voltage Validation ===\n");

    // TODO: Calculate and display average

    return 0;
}
\end{lstlisting}

\subsection{Task 4: Simple Calculator Functions (60 minutes)}

\textbf{File:} \texttt{calculator\_functions.c}

\textbf{Requirements:}
\begin{itemize}
    \item Create functions for basic mathematical operations
    \item Practice functions with multiple parameters
    \item Use return values to send results back to main()
    \item Build a simple menu-driven calculator
\end{itemize}

\textbf{Required Functions:}
\begin{lstlisting}[language=C]
#include <stdio.h>

// TODO: Create basic math functions
float add_numbers(float a, float b) {
    // TODO: Return a + b
}

float multiply_numbers(float a, float b) {
    // TODO: Return a * b
}

float calculate_power_consumption(float voltage, float current) {
    // TODO: Return voltage * current (P = V * I)
}

int is_safe_temperature(float temp) {
    // TODO: Return 1 if temp < 85.0, 0 otherwise
}

void print_menu() {
    // TODO: Print calculator menu options
}

int main() {
    // TODO: Create simple calculator with menu
    // TODO: Use functions to perform calculations
    // TODO: Display results clearly

    return 0;
}
\end{lstlisting}

\subsection{Task 5: Basic Debugging with Printf (30 minutes)}

\textbf{Requirements:}
\begin{itemize}
    \item Learn to use printf statements for debugging
    \item Practice finding and fixing simple bugs
    \item Understand how to trace program execution
    \item Document debugging process in simple terms
\end{itemize}

\textbf{Debugging Exercises:}
\begin{enumerate}
    \item Add printf statements to trace function calls
    \item Use printf to display variable values
    \item Find and fix a simple calculation error
    \item Practice reading compiler error messages
\end{enumerate}

\textbf{Simple Debugging Example:}
\begin{lstlisting}[language=C]
#include <stdio.h>

float calculate_power(float voltage, float current) {
    printf("DEBUG: voltage = %.2f, current = %.2f\n", voltage, current);
    float power = voltage * current;
    printf("DEBUG: calculated power = %.2f\n", power);
    return power;
}

int main() {
    printf("DEBUG: Starting program\n");
    float result = calculate_power(3.3, 0.5);
    printf("Final result: %.2fW\n", result);
    printf("DEBUG: Program finished\n");
    return 0;
}
\end{lstlisting}

\section{Part 2: Optional Extensions (For Advanced Students)}

\textbf{Note:} These tasks are completely optional and designed for students who finish early or want extra challenges. Focus on completing Part 1 first!

\subsection{Extension 1: Enhanced Error Handling (Optional)}

\textbf{For students who want to explore input validation:}
\begin{itemize}
    \item Add basic input validation for non-numeric entries
    \item Implement simple retry mechanism for invalid inputs
    \item Add helpful error messages
\end{itemize}

\textbf{Simple Example:}
\begin{lstlisting}[language=C]
#include <stdio.h>

int main() {
    float voltage;
    int valid_input = 0;

    while (!valid_input) {
        printf("Enter voltage: ");
        if (scanf("%f", &voltage) == 1) {
            valid_input = 1;
        } else {
            printf("Please enter a valid number.\n");
            // Clear input buffer
            while (getchar() != '\n');
        }
    }

    // Continue with validation...
    return 0;
}
\end{lstlisting}

\subsection{Extension 2: Multi-Parameter Validator (Optional)}

\textbf{File:} \texttt{multi\_validator.c}

\textbf{For students who want to combine concepts:}
\begin{itemize}
    \item Combine voltage, power, and temperature validation
    \item Create a simple scoring system
    \item Display overall chip health status
\end{itemize}

\subsection{Extension 3: Creative Enhancements (Optional)}

\textbf{Ideas for exploration:}
\begin{itemize}
    \item Add colorful output using ANSI codes
    \item Create ASCII art displays for results
    \item Add more chip parameters to validate
    \item Research real chip specifications and use them
\end{itemize}

\section{Submission Requirements}

\subsection{Code Quality Standards}
\begin{itemize}
    \item \textbf{Function Design:} Each function should have a single, clear purpose
    \item \textbf{Error Handling:} All functions should handle edge cases gracefully
    \item \textbf{Loop Efficiency:} Avoid infinite loops, optimize for performance
    \item \textbf{Variable Scope:} Use appropriate variable scope (local vs. global)
    \item \textbf{Magic Numbers:} Use \texttt{\#define} constants instead of hard-coded values
\end{itemize}

\subsection{Documentation Requirements}
\begin{itemize}
    \item \textbf{Function Documentation:} Comment each function's purpose, parameters, and return values
    \item \textbf{README.md:} Updated with Day 2 functionality and usage instructions
    \item \textbf{Simple comments:} Add comments explaining your logic in each program
    \item \textbf{Testing notes:} Document what test cases you tried
\end{itemize}

\section{Grading Rubric}

\begin{center}
\begin{tabular}{|l|c|l|}
\hline
\textbf{Component} & \textbf{Points} & \textbf{Criteria} \\
\hline
Task 1: Simple Functions & 25 & Function creation, calling, return values \\
Task 2: Loop Practice & 25 & Basic for/while loops, loop control \\
Task 3: Arrays \& Functions & 25 & Array processing, function parameters \\
Task 4: Calculator Functions & 20 & Multiple functions, menu system \\
Task 5: Printf Debugging & 15 & Debug output, problem solving \\
Code Quality & 15 & Comments, style, variable naming \\
Documentation & 10 & README, clear explanations \\
GitHub Usage & 10 & Proper commits, submission \\
\hline
\textbf{Subtotal} & \textbf{145} & \textbf{Base assignment} \\
\hline
Optional Extensions & +15 & Extra credit for advanced features \\
\hline
\textbf{Total Possible} & \textbf{160} & \textbf{With extra credit} \\
\hline
\end{tabular}
\end{center}

\section{Testing and Validation}

\subsection{Required Test Cases}
\begin{enumerate}
    \item \textbf{Function Tests:} Test each function with sample inputs
    \item \textbf{Loop Tests:} Verify loops count correctly and terminate properly
    \item \textbf{Array Tests:} Check array processing with different values
    \item \textbf{Calculator Tests:} Test mathematical operations with various inputs
    \item \textbf{Edge Cases:} Test with boundary values (0, negative numbers, etc.)
\end{enumerate}

\subsection{Simple Testing Guidelines}
\begin{itemize}
    \item Test each function as you write it
    \item Use printf to verify intermediate results
    \item Try different input values to ensure correctness
    \item Make sure your programs compile without warnings
\end{itemize}

\section{Common Challenges and Solutions}

\subsection{Infinite Loops}
\textbf{Problem:} While loops that never terminate
\textbf{Solution:} Always ensure loop conditions can become false, add safety counters

\subsection{Off-by-One Errors}
\textbf{Problem:} Array bounds violations in loops
\textbf{Solution:} Carefully check loop conditions, use < instead of <= when appropriate

\subsection{Function Parameter Issues}
\textbf{Problem:} Incorrect parameter passing or return values
\textbf{Solution:} Use GDB to trace parameter values, add parameter validation

\subsection{Complex Conditional Logic}
\textbf{Problem:} Nested if-else statements become unreadable
\textbf{Solution:} Break into separate functions, use early returns, consider switch statements

\section{Extension Opportunities}

\subsection{Advanced Features}
\begin{itemize}
    \item \textbf{State Machine Testing:} Implement register state transitions
    \item \textbf{Concurrent Testing:} Simulate multi-threaded register access
    \item \textbf{Data-Driven Tests:} Read test configurations from files
    \item \textbf{Visual Output:} ASCII charts showing test progress
    \item \textbf{Test Scheduling:} Priority-based test execution
\end{itemize}

\subsection{Real-World Applications}
\begin{itemize}
    \item Research actual register testing in semiconductor validation
    \item Investigate commercial register testing tools
    \item Study fault injection techniques used in industry
    \item Explore automated test equipment (ATE) programming
\end{itemize}

\section{Success Tips}

\begin{itemize}
    \item \textbf{Start Simple:} Begin with basic loops and conditions, add complexity gradually
    \item \textbf{Test Incrementally:} Test each function as you write it
    \item \textbf{Use GDB Effectively:} Don't guess at bugs, use the debugger to understand them
    \item \textbf{Plan Your Functions:} Think about the interface before implementing
    \item \textbf{Handle Errors Early:} Add error checking as you write code, not as an afterthought
    \item \textbf{Document as You Go:} Write comments and documentation while the logic is fresh
\end{itemize}

\vspace{1cm}

\begin{center}
\textbf{Master the art of systematic testing and debugging!}\\
\textit{These skills are the foundation of reliable validation systems.}
\end{center}

\end{document}

