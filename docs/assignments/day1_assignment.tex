\documentclass[11pt,a4paper]{article}
\usepackage[utf8]{inputenc}
\usepackage[T1]{fontenc}
\usepackage{geometry}
\usepackage{graphicx}
\usepackage{xcolor}
\usepackage{listings}
\usepackage{fancyhdr}
\usepackage{titlesec}
\usepackage{hyperref}
\usepackage{enumitem}
\usepackage{booktabs}
\usepackage{array}

% Page setup
\geometry{margin=1in}
\pagestyle{fancy}
\fancyhf{}
\fancyhead[L]{Day 1 Assignment: C Fundamentals Lab}
\fancyhead[R]{\thepage}
\fancyfoot[C]{C Programming for Post-Silicon Validation Engineers}

% Colors
\definecolor{codeblue}{RGB}{0,102,204}
\definecolor{codegray}{RGB}{128,128,128}
\definecolor{codegreen}{RGB}{0,128,0}
\definecolor{backcolour}{RGB}{245,245,245}

% Code listing style
\lstdefinestyle{cstyle}{
    backgroundcolor=\color{backcolour},
    commentstyle=\color{codegreen},
    keywordstyle=\color{codeblue},
    numberstyle=\tiny\color{codegray},
    stringstyle=\color{red},
    basicstyle=\ttfamily\footnotesize,
    breakatwhitespace=false,
    breaklines=true,
    captionpos=b,
    keepspaces=true,
    numbers=left,
    numbersep=5pt,
    showspaces=false,
    showstringspaces=false,
    showtabs=false,
    tabsize=2,
    frame=single
}

\lstset{style=cstyle}

% Title formatting
\titleformat{\section}{\Large\bfseries\color{codeblue}}{\thesection}{1em}{}
\titleformat{\subsection}{\large\bfseries}{\thesubsection}{1em}{}

\hypersetup{
    colorlinks=true,
    linkcolor=codeblue,
    filecolor=magenta,
    urlcolor=cyan,
    pdftitle={Day 1 Assignment - C Fundamentals Lab},
    pdfauthor={Yahwista Salomo},
}

\begin{document}

% Header
\begin{center}
    {\Huge\bfseries\color{codeblue} DAY 1 ASSIGNMENT}\\[0.5cm]
    {\Large C Fundamentals and Compilation Lab}\\[0.3cm]
    {\large Chip Parameter Validation System}\\[0.2cm]
    {\normalsize Due: End of Day 1 + Homework Extension}
\end{center}

\vspace{1cm}

\section{Assignment Overview}

\subsection{Learning Objectives}
By completing this assignment, you will:
\begin{itemize}
    \item Write your first C programs using variables, operators, and basic I/O
    \item Learn to compile programs using GCC
    \item Practice basic input/output with \texttt{printf} and \texttt{scanf}
    \item Understand basic conditional statements and comparisons
    \item Create simple validation programs relevant to post-silicon testing
    \item Submit work using GitHub Classroom (Git basics covered in previous session)
\end{itemize}

\subsection{Prerequisites}
Before starting this assignment, you should:
\begin{itemize}
    \item Have GCC compiler installed and tested (see Setup Guide)
    \item Have a text editor or IDE ready (VS Code, CLion, or simple text editor)
    \item Have Git configured (covered in previous session)
    \item Understand basic mathematical operations (+, -, *, /)
    \item Be comfortable with basic command line navigation
\end{itemize}

\subsection{Assignment Context}
You will build simple programs that check basic parameters, similar to what engineers do in chip validation. We'll start very simple with just checking if a number is in a range, then gradually add more features. The focus is on learning C fundamentals through practical, engineering-relevant examples.

\section{Part 1: In-Class Lab (3.5 hours)}

\subsection{Setup and Repository}
\begin{enumerate}
    \item Accept the GitHub Classroom assignment invitation (link provided in class)
    \item Clone your personal repository with starter code: \texttt{git clone [your-repo-url]}
    \item Navigate to the project directory: \texttt{cd day1-assignment-[your-username]}
    \item Verify GCC installation: \texttt{gcc --version}
    \item Explore the starter files provided in your repository
\end{enumerate}

\textbf{Note:} Your repository comes with starter code templates and example files to help you get started!

\subsection{Task 1: Basic Voltage Validator (60 minutes)}

\textbf{File:} \texttt{voltage\_validator.c}

\textbf{Requirements:}
\begin{itemize}
    \item Prompt user for supply voltage input
    \item Read the voltage using \texttt{scanf}
    \item Check if voltage is within acceptable range (1.8V - 3.6V)
    \item Display clear PASS/FAIL results
    \item Use simple if-else statements for validation
\end{itemize}

\textbf{Starter Code Template:}
\begin{lstlisting}[language=C]
#include <stdio.h>

int main() {
    // TODO: Declare variables for voltage and limits
    float voltage;
    float min_voltage = 1.8;
    float max_voltage = 3.6;

    // TODO: Display program header and instructions
    printf("=== Chip Voltage Validator ===\n");

    // TODO: Get voltage input from user
    printf("Enter supply voltage (V): ");
    scanf("%f", &voltage);

    // TODO: Validate voltage range
    if (voltage >= min_voltage && voltage <= max_voltage) {
        // TODO: Display PASS message
    } else {
        // TODO: Display FAIL message
    }

    return 0;
}
\end{lstlisting}

\textbf{Expected Output Example:}
\begin{verbatim}
=== Chip Voltage Validator ===
Enter supply voltage (V): 3.3
Voltage: 3.30V
Valid range: 1.8V - 3.6V
Result: PASS - Voltage within acceptable range
\end{verbatim}

\textbf{Step-by-Step Guide:}
\begin{enumerate}
    \item Complete the TODO comments in the starter code
    \item Test with the example input (3.3V) - should show PASS
    \item Test with invalid input (1.0V) - should show FAIL
    \item Make sure your output matches the expected format
    \item Compile and fix any errors before moving to Task 2
\end{enumerate}

\textbf{Step-by-Step Guide:}
\begin{enumerate}
    \item Complete the TODO comments in the starter code
    \item Add printf statements for PASS and FAIL cases
    \item Test with values like 2.5V (should pass) and 4.0V (should fail)
    \item Make sure your output matches the expected format
\end{enumerate}

\textbf{Helpful Tips:}
\begin{itemize}
    \item Use \texttt{\%.2f} in printf to display 2 decimal places
    \item The \texttt{\&} symbol in scanf is required for reading variables
    \item Test your program multiple times with different values
    \item Don't worry about advanced error handling for now - focus on the basic logic
\end{itemize}

\subsection{Task 2: Power Consumption Calculator (45 minutes)}

\textbf{File:} \texttt{power\_calculator.c}

\textbf{Requirements:}
\begin{itemize}
    \item Input: voltage and current values
    \item Calculate power consumption (P = V × I)
    \item Validate power is within acceptable limits (< 5W)
    \item Display power efficiency rating
\end{itemize}

\textbf{Power Efficiency Ratings:}
\begin{itemize}
    \item Excellent: < 1W
    \item Good: 1W - 2W
    \item Acceptable: 2W - 3W
    \item High: 3W - 5W
    \item Excessive: > 5W (FAIL)
\end{itemize}

\subsection{Task 3: Simple Temperature Checker (45 minutes)}

\textbf{File:} \texttt{temperature\_checker.c}

\textbf{Requirements:}
\begin{itemize}
    \item Input: chip temperature in Celsius
    \item Check if temperature is safe (< 85°C)
    \item Display temperature status with clear messages
    \item Use simple if-else logic for validation
\end{itemize}

\textbf{Temperature Status Messages:}
\begin{itemize}
    \item Cool: < 50°C - "Chip running cool"
    \item Normal: 50°C - 70°C - "Normal operating temperature"
    \item Warm: 70°C - 85°C - "Running warm but acceptable"
    \item Hot: > 85°C - "FAIL: Temperature too high!"
\end{itemize}

\subsection{Task 4: Compilation and Testing (30 minutes)}

\textbf{Compilation Requirements:}
\begin{itemize}
    \item Compile each program with: \texttt{gcc -Wall -g -std=c11 program.c -o program}
    \item Fix all compiler warnings
    \item Test with various input values
    \item Document any issues in \texttt{TESTING.md}
\end{itemize}

\textbf{Test Cases to Verify:}
\begin{enumerate}
    \item Valid inputs within all ranges
    \item Boundary values (exactly at limits)
    \item Invalid inputs (out of range)
    \item Edge cases (negative values, zero)
    \item Non-numeric input handling
\end{enumerate}

\section{Part 2: Optional Extensions (For Advanced Students)}

\textbf{Note:} These tasks are completely optional and designed for students who finish early or want extra challenges. Focus on completing Part 1 first!

\subsection{Extension 1: Enhanced Error Handling (Optional)}

\textbf{For students who want to explore input validation:}
\begin{itemize}
    \item Add basic input validation for non-numeric entries
    \item Implement simple retry mechanism for invalid inputs
    \item Add helpful error messages
\end{itemize}

\textbf{Simple Example:}
\begin{lstlisting}[language=C]
#include <stdio.h>

int main() {
    float voltage;
    int valid_input = 0;

    while (!valid_input) {
        printf("Enter supply voltage (1.8V - 3.6V): ");

        if (scanf("%f", &voltage) == 1) {
            if (voltage >= 1.8 && voltage <= 3.6) {
                valid_input = 1;
                printf("Valid voltage: %.2fV\n", voltage);
            } else {
                printf("Error: Voltage must be between 1.8V and 3.6V\n");
            }
        } else {
            printf("Error: Please enter a valid number\n");
            // Clear input buffer
            while (getchar() != '\n');
        }
    }

    return 0;
}
\end{lstlisting}

\subsection{Extension 2: Multi-Parameter Validator (Optional)}

\textbf{For students who want to combine multiple checks:}
\begin{itemize}
    \item Combine voltage, power, and temperature validation in one program
    \item Calculate an overall "health score" for the chip
    \item Display a summary report
\end{itemize}

\subsection{Extension 3: Creative Enhancements (Optional)}

\textbf{Ideas for creative students:}
\begin{itemize}
    \item Add ASCII art for PASS/FAIL results
    \item Create a simple menu system
    \item Add color output (if your terminal supports it)
    \item Implement a simple logging system
    \item Add unit conversion (Fahrenheit to Celsius, etc.)
\end{itemize}

\section{Submission Requirements}

\subsection{Code Quality Standards}
\begin{itemize}
    \item \textbf{Compilation:} All programs must compile without warnings
    \item \textbf{Comments:} Clear, descriptive comments for all major sections
    \item \textbf{Variable Names:} Descriptive names (e.g., \texttt{supply\_voltage} not \texttt{v})
    \item \textbf{Formatting:} Consistent indentation and spacing
    \item \textbf{Error Handling:} Robust input validation and error messages
\end{itemize}

\subsection{Documentation Requirements}
\begin{itemize}
    \item \textbf{README.md:} Project description, compilation instructions, usage examples
    \item \textbf{TESTING.md:} Test cases, results, and any issues encountered
    \item \textbf{Code Comments:} Inline documentation explaining logic
\end{itemize}

\subsection{GitHub Submission}
\begin{enumerate}
    \item Commit your work regularly with descriptive messages
    \item Push all changes to your GitHub repository
    \item Create a pull request with title "Day 1 Assignment Submission"
    \item Include a summary of completed tasks in the PR description
    \item Submit the PR URL via the course submission form
\end{enumerate}

\section{Grading Rubric}

\begin{center}
\begin{tabular}{|l|c|l|}
\hline
\textbf{Component} & \textbf{Points} & \textbf{Criteria} \\
\hline
Task 1: Voltage Validator & 25 & Correct logic, proper I/O, compilation \\
Task 2: Power Calculator & 25 & Accurate calculations, validation logic \\
Task 3: Temperature Checker & 25 & Temperature ranges, clear output \\
Task 4: Compilation \& Testing & 15 & Clean compilation, proper testing \\
Code Quality & 15 & Comments, style, variable naming \\
Documentation & 10 & README, clear explanations \\
GitHub Usage & 10 & Proper commits, submission \\
\hline
\textbf{Subtotal} & \textbf{125} & \textbf{Base assignment} \\
\hline
Optional Extensions & +15 & Extra credit for advanced features \\
\hline
\textbf{Total Possible} & \textbf{140} & \textbf{With extra credit} \\
\hline
\end{tabular}
\end{center}

\subsection{Grading Scale}
\begin{itemize}
    \item \textbf{A (90-100\%):} Exceptional work, all requirements met plus enhancements
    \item \textbf{B (80-89\%):} Good work, most requirements met with minor issues
    \item \textbf{C (70-79\%):} Satisfactory work, basic requirements met
    \item \textbf{D (60-69\%):} Below expectations, significant issues or missing components
    \item \textbf{F (<60\%):} Unsatisfactory, major requirements not met
\end{itemize}

\section{Getting Help}

\subsection{During Lab Time}
\begin{itemize}
    \item Raise your hand for instructor or TA assistance
    \item Work with your assigned pair programming partner
    \item Use the class discussion board for questions
    \item Collaborate with nearby teams (but submit individual work)
\end{itemize}

\subsection{For Homework}
\begin{itemize}
    \item Review course materials and lecture slides
    \item Use online C programming resources (document sources)
\end{itemize}

\subsection{Common Issues and Solutions}
\begin{itemize}
    \item \textbf{Compilation errors:} Check syntax, missing semicolons, header includes
    \item \textbf{Input issues:} Use \texttt{scanf} return value checking, clear input buffer
    \item \textbf{Logic errors:} Use printf debugging, trace through code manually
    \item \textbf{Git issues:} Check file staging, commit messages, push status
\end{itemize}

\section{Extension Opportunities}

\subsection{Advanced Features (Extra Credit)}
\begin{itemize}
    \item \textbf{Graphical Output:} ASCII charts showing parameter ranges
    \item \textbf{Data Logging:} Timestamp and log all validation results
    \item \textbf{Statistical Analysis:} Calculate mean, standard deviation of measurements
    \item \textbf{Multi-Language Support:} Internationalization for error messages
    \item \textbf{Unit Testing:} Automated test suite for validation functions
\end{itemize}

\subsection{Real-World Connections}
\begin{itemize}
    \item Research actual MicroBlaze-V specifications and compare to your limits
    \item Investigate post-silicon validation practices in industry
    \item Explore how validation tools are used in semiconductor companies
    \item Connect with professionals in validation engineering roles
\end{itemize}

\section{Success Tips}

\begin{itemize}
    \item \textbf{Start Early:} Begin with the basic tasks and build complexity gradually
    \item \textbf{Test Frequently:} Compile and test after each small change
    \item \textbf{Read Carefully:} Pay attention to requirements and expected outputs
    \item \textbf{Ask Questions:} Don't hesitate to seek help when stuck
    \item \textbf{Document Everything:} Good documentation helps with debugging and grading
    \item \textbf{Have Fun:} Enjoy the process of creating your first validation tools!
\end{itemize}

\vspace{1cm}

\begin{center}
\textbf{Good luck with your first C programming assignment!}\\
\textit{You're building the foundation for your validation engineering career.}
\end{center}

\end{document}

