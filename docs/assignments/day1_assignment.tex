\documentclass[11pt,a4paper]{article}
\usepackage[utf8]{inputenc}
\usepackage[T1]{fontenc}
\usepackage{geometry}
\usepackage{graphicx}
\usepackage{xcolor}
\usepackage{listings}
\usepackage{fancyhdr}
\usepackage{titlesec}
\usepackage{hyperref}
\usepackage{enumitem}
\usepackage{booktabs}
\usepackage{array}

% Page setup
\geometry{margin=1in}
\pagestyle{fancy}
\fancyhf{}
\fancyhead[L]{Day 1 Assignment: C Fundamentals Lab}
\fancyhead[R]{\thepage}
\fancyfoot[C]{C Programming for Post-Silicon Validation Engineers}

% Colors
\definecolor{codeblue}{RGB}{0,102,204}
\definecolor{codegray}{RGB}{128,128,128}
\definecolor{codegreen}{RGB}{0,128,0}
\definecolor{backcolour}{RGB}{245,245,245}

% Code listing style
\lstdefinestyle{cstyle}{
    backgroundcolor=\color{backcolour},
    commentstyle=\color{codegreen},
    keywordstyle=\color{codeblue},
    numberstyle=\tiny\color{codegray},
    stringstyle=\color{red},
    basicstyle=\ttfamily\footnotesize,
    breakatwhitespace=false,
    breaklines=true,
    captionpos=b,
    keepspaces=true,
    numbers=left,
    numbersep=5pt,
    showspaces=false,
    showstringspaces=false,
    showtabs=false,
    tabsize=2,
    frame=single
}

\lstset{style=cstyle}

% Title formatting
\titleformat{\section}{\Large\bfseries\color{codeblue}}{\thesection}{1em}{}
\titleformat{\subsection}{\large\bfseries}{\thesubsection}{1em}{}

\hypersetup{
    colorlinks=true,
    linkcolor=codeblue,
    filecolor=magenta,
    urlcolor=cyan,
    pdftitle={Day 1 Assignment - C Fundamentals Lab},
    pdfauthor={Yahwista Salomo},
}

\begin{document}

% Header
\begin{center}
    {\Huge\bfseries\color{codeblue} DAY 1 ASSIGNMENT}\\[0.5cm]
    {\Large C Fundamentals and Compilation Lab}\\[0.3cm]
    {\large Chip Parameter Validation System}\\[0.2cm]
    {\normalsize Due: End of Day 1 + Homework Extension}
\end{center}

\vspace{1cm}

\section{Assignment Overview}

\subsection{Learning Objectives}
By completing this assignment, you will:
\begin{itemize}
    \item Write C programs using variables, operators, and basic I/O
    \item Compile programs using GCC with appropriate flags
    \item Handle user input validation and error checking
    \item Create validation programs relevant to post-silicon testing
    \item Use GitHub for version control and submission
\end{itemize}

\subsection{Assignment Context}
You will build a chip parameter validation system that simulates real-world post-silicon validation scenarios. This system will check voltage levels, calculate power consumption, and validate operating parameters against specifications.

\section{Part 1: In-Class Lab (3.5 hours)}

\subsection{Setup and Repository}
\begin{enumerate}
    \item Accept the GitHub Classroom assignment invitation
    \item Clone your personal repository: \texttt{git clone [your-repo-url]}
    \item Navigate to the project directory
    \item Verify GCC installation: \texttt{gcc --version}
\end{enumerate}

\subsection{Task 1: Basic Voltage Validator (45 minutes)}

\textbf{File:} \texttt{voltage\_validator.c}

\textbf{Requirements:}
\begin{itemize}
    \item Prompt user for supply voltage input
    \item Validate voltage is within acceptable range (1.8V - 3.6V)
    \item Display clear PASS/FAIL results with explanations
    \item Handle invalid input gracefully
\end{itemize}

\textbf{Starter Code Template:}
\begin{lstlisting}[language=C]
#include <stdio.h>

int main() {
    // TODO: Declare variables for voltage and limits

    // TODO: Display program header and instructions

    // TODO: Get voltage input from user

    // TODO: Validate voltage range

    // TODO: Display results with clear messaging

    return 0;
}
\end{lstlisting}

\textbf{Expected Output Example:}
\begin{verbatim}
=== Chip Voltage Validator ===
Enter supply voltage (V): 3.3
Voltage: 3.30V
Valid range: 1.8V - 3.6V
Result: PASS - Voltage within acceptable range
\end{verbatim}

\subsection{Task 2: Power Consumption Calculator (45 minutes)}

\textbf{File:} \texttt{power\_calculator.c}

\textbf{Requirements:}
\begin{itemize}
    \item Input: voltage and current values
    \item Calculate power consumption (P = V × I)
    \item Validate power is within acceptable limits (< 5W)
    \item Display power efficiency rating
\end{itemize}

\textbf{Power Efficiency Ratings:}
\begin{itemize}
    \item Excellent: < 1W
    \item Good: 1W - 2W
    \item Acceptable: 2W - 3W
    \item High: 3W - 5W
    \item Excessive: > 5W (FAIL)
\end{itemize}

\subsection{Task 3: Multi-Parameter Validator (60 minutes)}

\textbf{File:} \texttt{chip\_validator.c}

\textbf{Requirements:}
\begin{itemize}
    \item Combine voltage and power validation
    \item Add temperature validation (< 85°C)
    \item Add clock frequency validation (1MHz - 133MHz for RP2040)
    \item Provide comprehensive pass/fail summary
    \item Calculate and display overall system health score
\end{itemize}

\textbf{Health Score Calculation:}
\begin{itemize}
    \item Each parameter: 25 points if PASS, 0 if FAIL
    \item Total possible: 100 points
    \item Display percentage and grade (A: 90-100, B: 80-89, C: 70-79, F: <70)
\end{itemize}

\subsection{Task 4: Compilation and Testing (30 minutes)}

\textbf{Compilation Requirements:}
\begin{itemize}
    \item Compile each program with: \texttt{gcc -Wall -g -std=c11 program.c -o program}
    \item Fix all compiler warnings
    \item Test with various input values
    \item Document any issues in \texttt{TESTING.md}
\end{itemize}

\textbf{Test Cases to Verify:}
\begin{enumerate}
    \item Valid inputs within all ranges
    \item Boundary values (exactly at limits)
    \item Invalid inputs (out of range)
    \item Edge cases (negative values, zero)
    \item Non-numeric input handling
\end{enumerate}

\section{Part 2: Homework Extension (2 hours)}

\subsection{Task 5: Enhanced Error Handling}

\textbf{Requirements:}
\begin{itemize}
    \item Add input validation for non-numeric entries
    \item Implement retry mechanism for invalid inputs
    \item Add range checking with specific error messages
    \item Create user-friendly error recovery
\end{itemize}

\textbf{Example Enhanced Error Handling:}
\begin{lstlisting}[language=C]
#include <stdio.h>

int get_valid_voltage(float *voltage) {
    int attempts = 0;
    const int max_attempts = 3;

    while (attempts < max_attempts) {
        printf("Enter supply voltage (1.8V - 3.6V): ");

        if (scanf("%f", voltage) != 1) {
            printf("Error: Please enter a valid number.\n");
            // Clear input buffer
            while (getchar() != '\n');
        } else if (*voltage < 1.8 || *voltage > 3.6) {
            printf("Error: Voltage %.2fV is out of range.\n", *voltage);
        } else {
            return 1; // Success
        }

        attempts++;
        printf("Attempt %d of %d failed. ", attempts, max_attempts);
        if (attempts < max_attempts) {
            printf("Please try again.\n");
        }
    }

    printf("Maximum attempts exceeded.\n");
    return 0; // Failure
}
\end{lstlisting}

\subsection{Task 6: Configuration File Support}

\textbf{File:} \texttt{config\_validator.c}

\textbf{Requirements:}
\begin{itemize}
    \item Read validation parameters from \texttt{config.txt}
    \item Support customizable voltage, power, and temperature limits
    \item Allow different chip profiles (RP2040, STM32, etc.)
    \item Implement configuration file parsing
\end{itemize}

\textbf{Sample config.txt:}
\begin{verbatim}
# Chip Validation Configuration
CHIP_TYPE=RP2040
MIN_VOLTAGE=1.8
MAX_VOLTAGE=3.6
MAX_POWER=5.0
MAX_TEMPERATURE=85.0
MIN_FREQUENCY=1000000
MAX_FREQUENCY=133000000
\end{verbatim}

\subsection{Task 7: Batch Testing Mode}

\textbf{File:} \texttt{batch\_validator.c}

\textbf{Requirements:}
\begin{itemize}
    \item Read test data from \texttt{test\_data.csv}
    \item Process multiple chip measurements automatically
    \item Generate summary statistics (pass rate, average values)
    \item Output results to \texttt{results.txt}
\end{itemize}

\section{Submission Requirements}

\subsection{Code Quality Standards}
\begin{itemize}
    \item \textbf{Compilation:} All programs must compile without warnings
    \item \textbf{Comments:} Clear, descriptive comments for all major sections
    \item \textbf{Variable Names:} Descriptive names (e.g., \texttt{supply\_voltage} not \texttt{v})
    \item \textbf{Formatting:} Consistent indentation and spacing
    \item \textbf{Error Handling:} Robust input validation and error messages
\end{itemize}

\subsection{Documentation Requirements}
\begin{itemize}
    \item \textbf{README.md:} Project description, compilation instructions, usage examples
    \item \textbf{TESTING.md:} Test cases, results, and any issues encountered
    \item \textbf{Code Comments:} Inline documentation explaining logic
\end{itemize}

\subsection{GitHub Submission}
\begin{enumerate}
    \item Commit your work regularly with descriptive messages
    \item Push all changes to your GitHub repository
    \item Create a pull request with title "Day 1 Assignment Submission"
    \item Include a summary of completed tasks in the PR description
    \item Submit the PR URL via the course submission form
\end{enumerate}

\section{Grading Rubric}

\begin{center}
\begin{tabular}{|l|c|l|}
\hline
\textbf{Component} & \textbf{Points} & \textbf{Criteria} \\
\hline
Task 1: Basic Validator & 15 & Correct logic, proper I/O, compilation \\
Task 2: Power Calculator & 15 & Accurate calculations, validation logic \\
Task 3: Multi-Parameter & 20 & Integration, health scoring, comprehensive testing \\
Task 4: Compilation & 10 & Clean compilation, proper GCC usage \\
Task 5: Error Handling & 15 & Robust input validation, user experience \\
Task 6: Configuration & 10 & File parsing, parameter customization \\
Task 7: Batch Processing & 10 & Automation, statistics, file I/O \\
Code Quality & 15 & Comments, style, variable naming \\
Documentation & 10 & README, testing docs, clarity \\
GitHub Usage & 5 & Commit quality, PR submission \\
\hline
\textbf{Total} & \textbf{125} & \textbf{Extra credit available for advanced features} \\
\hline
\end{tabular}
\end{center}

\subsection{Grading Scale}
\begin{itemize}
    \item \textbf{A (90-100\%):} Exceptional work, all requirements met plus enhancements
    \item \textbf{B (80-89\%):} Good work, most requirements met with minor issues
    \item \textbf{C (70-79\%):} Satisfactory work, basic requirements met
    \item \textbf{D (60-69\%):} Below expectations, significant issues or missing components
    \item \textbf{F (<60\%):} Unsatisfactory, major requirements not met
\end{itemize}

\section{Getting Help}

\subsection{During Lab Time}
\begin{itemize}
    \item Raise your hand for instructor or TA assistance
    \item Work with your assigned pair programming partner
    \item Use the class discussion board for questions
    \item Collaborate with nearby teams (but submit individual work)
\end{itemize}

\subsection{For Homework}
\begin{itemize}
    \item Post questions in the course Slack channel
    \item Attend virtual office hours (schedule TBD)
    \item Review course materials and lecture slides
    \item Use online C programming resources (document sources)
\end{itemize}

\subsection{Common Issues and Solutions}
\begin{itemize}
    \item \textbf{Compilation errors:} Check syntax, missing semicolons, header includes
    \item \textbf{Input issues:} Use \texttt{scanf} return value checking, clear input buffer
    \item \textbf{Logic errors:} Use printf debugging, trace through code manually
    \item \textbf{Git issues:} Check file staging, commit messages, push status
\end{itemize}

\section{Extension Opportunities}

\subsection{Advanced Features (Extra Credit)}
\begin{itemize}
    \item \textbf{Graphical Output:} ASCII charts showing parameter ranges
    \item \textbf{Data Logging:} Timestamp and log all validation results
    \item \textbf{Statistical Analysis:} Calculate mean, standard deviation of measurements
    \item \textbf{Multi-Language Support:} Internationalization for error messages
    \item \textbf{Unit Testing:} Automated test suite for validation functions
\end{itemize}

\subsection{Real-World Connections}
\begin{itemize}
    \item Research actual RP2040 specifications and compare to your limits
    \item Investigate post-silicon validation practices in industry
    \item Explore how validation tools are used in semiconductor companies
    \item Connect with professionals in validation engineering roles
\end{itemize}

\section{Success Tips}

\begin{itemize}
    \item \textbf{Start Early:} Begin with the basic tasks and build complexity gradually
    \item \textbf{Test Frequently:} Compile and test after each small change
    \item \textbf{Read Carefully:} Pay attention to requirements and expected outputs
    \item \textbf{Ask Questions:} Don't hesitate to seek help when stuck
    \item \textbf{Document Everything:} Good documentation helps with debugging and grading
    \item \textbf{Have Fun:} Enjoy the process of creating your first validation tools!
\end{itemize}

\vspace{1cm}

\begin{center}
\textbf{Good luck with your first C programming assignment!}\\
\textit{You're building the foundation for your validation engineering career.}
\end{center}

\end{document}

