\documentclass[11pt,a4paper]{article}
\usepackage[utf8]{inputenc}
\usepackage[T1]{fontenc}
\usepackage{geometry}
\usepackage{graphicx}
\usepackage{xcolor}
\usepackage{listings}
\usepackage{fancyhdr}
\usepackage{titlesec}
\usepackage{hyperref}
\usepackage{enumitem}
\usepackage{booktabs}
\usepackage{array}
\usepackage{amssymb}

% Page setup
\geometry{margin=1in}
\setlength{\headheight}{14pt}
\pagestyle{fancy}
\fancyhf{}
\fancyhead[L]{Day 3 Assignment: Memory Management and Data Structures Lab}
\fancyhead[R]{\thepage}
\fancyfoot[C]{C Programming for Post-Silicon Validation Engineers}

% Colors
\definecolor{codeblue}{RGB}{0,102,204}
\definecolor{codegray}{RGB}{128,128,128}
\definecolor{codegreen}{RGB}{0,128,0}
\definecolor{backcolour}{RGB}{245,245,245}

% Code listing style
\lstdefinestyle{cstyle}{
    backgroundcolor=\color{backcolour},
    commentstyle=\color{codegreen},
    keywordstyle=\color{codeblue},
    numberstyle=\tiny\color{codegray},
    stringstyle=\color{red},
    basicstyle=\ttfamily\footnotesize,
    breakatwhitespace=false,
    breaklines=true,
    captionpos=b,
    keepspaces=true,
    numbers=left,
    numbersep=5pt,
    showspaces=false,
    showstringspaces=false,
    showtabs=false,
    tabsize=2,
    frame=single
}

\lstset{style=cstyle}

% Title formatting
\titleformat{\section}{\Large\bfseries\color{codeblue}}{\thesection}{1em}{}
\titleformat{\subsection}{\large\bfseries}{\thesubsection}{1em}{}

\hypersetup{
    colorlinks=true,
    linkcolor=codeblue,
    filecolor=magenta,
    urlcolor=cyan,
    pdftitle={Day 3 Assignment - Memory Management and Data Structures Lab},
    pdfauthor={Yahwista Salomo},
}

\begin{document}

% Header
\begin{center}
    {\Huge\bfseries\color{codeblue} DAY 3 ASSIGNMENT}\\[0.5cm]
    {\Large Memory Management and Data Structures Lab}\\[0.3cm]
    {\large Chip State Monitoring and Bit Manipulation}\\[0.2cm]
    {\normalsize Due: End of Day 3 + Homework Extension}
\end{center}

\vspace{1cm}

\section{Assignment Overview}

\subsection{Learning Objectives}
By completing this assignment, you will:
\begin{itemize}
    \item Learn basic string handling and manipulation in C
    \item Understand and use simple structures to organize data
    \item Practice basic file input and output operations
    \item Implement simple error handling and validation
    \item Build programs that combine strings, structures, and file I/O
\end{itemize}

\subsection{Prerequisites}
Before starting this assignment, you should have completed Day 1-2 and be comfortable with:
\begin{itemize}
    \item Basic C syntax, variables, and I/O operations
    \item Simple functions with parameters and return values
    \item Basic loops (for and while) and arrays
    \item Compiling and testing C programs
    \item Basic Git operations for version control
\end{itemize}

\subsection{Assignment Context}
You will build simple programs that work with text data and organize information using structures. We'll start with basic string operations, then learn to group related data using structures, and finally add file operations to save and load data. The focus is on practical data handling skills used in engineering applications.

\section{Part 1: In-Class Lab (3 hours)}

\subsection{Setup and Repository}
\begin{enumerate}
    \item Accept the Day 3 GitHub Classroom assignment (link provided in class)
    \item Clone your personal repository: \texttt{git clone [your-repo-url]}
    \item Navigate to the project directory: \texttt{cd day3-assignment-[username]}
    \item Verify your Day 1-2 work is committed and pushed
    \item Create a new branch: \texttt{git checkout -b day3-strings-structures}
\end{enumerate}

\textbf{Note:} Your repository comes with starter code templates and example files to help you get started!

\subsection{Task 1: Basic String Handling (45 minutes)}

\textbf{File:} \texttt{string\_basics.c}

\textbf{Requirements:}
\begin{itemize}
    \item Learn to work with strings and character arrays
    \item Practice basic string functions like strlen and strcpy
    \item Create simple string processing functions
    \item Handle chip names and part numbers as strings
\end{itemize}

\textbf{Starter Code Template:}
\begin{lstlisting}[language=C]
#include <stdio.h>
#include <string.h>

// TODO: Create a function to display chip information
void display_chip_info(char chip_name[], char part_number[]) {
    // TODO: Print chip name and part number nicely formatted
}

// TODO: Create a function to check if chip name is valid
int is_valid_chip_name(char name[]) {
    // TODO: Return 1 if name length is between 3 and 15 characters, 0 otherwise
}

// TODO: Create a function to create a chip ID from name and number
void create_chip_id(char name[], char part[], char result[]) {
    // TODO: Combine name and part with a dash (e.g., "STM32-F407")
}

int main() {
    // TODO: Declare string variables for chip information
    char chip_name[20];
    char part_number[20];
    char chip_id[40];

    // TODO: Get chip information from user
    printf("Enter chip name: ");
    scanf("%s", chip_name);

    printf("Enter part number: ");
    scanf("%s", part_number);

    // TODO: Validate and display information

    return 0;
}
\end{lstlisting}

\textbf{Expected Output Example:}
\begin{verbatim}
Enter chip name: STM32
Enter part number: F407
=== Chip Information ===
Chip Name: STM32
Part Number: F407
Full Chip ID: STM32-F407
Name Length: 5 characters
Status: Valid chip name
\end{verbatim}

\subsection{Task 2: Simple Chip Structures (60 minutes)}

\textbf{File:} \texttt{chip\_structures.c}

\textbf{Requirements:}
\begin{itemize}
    \item Learn to define simple structures to group related data
    \item Practice initializing and using structure members
    \item Create functions that work with structures
    \item Understand how structures help organize information
\end{itemize}

\textbf{Simple Structure Definition:}
\begin{lstlisting}[language=C]
#include <stdio.h>
#include <string.h>

// TODO: Define a simple chip structure
typedef struct {
    char name[20];
    char part_number[20];
    float voltage;
    float temperature;
    int is_working;  // 1 for working, 0 for not working
} chip_info_t;

// TODO: Create a function to initialize a chip structure
void init_chip(chip_info_t* chip, char name[], char part[], float voltage) {
    // TODO: Copy name and part number, set voltage, set temperature to 0
    // TODO: Set is_working to 1 (working)
}

// TODO: Create a function to display chip information
void display_chip(chip_info_t* chip) {
    // TODO: Print all chip information in a nice format
}

// TODO: Create a function to check if chip is safe
int is_chip_safe(chip_info_t* chip) {
    // TODO: Return 1 if voltage is 1.8-3.6V and temp < 85°C, 0 otherwise
}

int main() {
    // TODO: Create a chip structure variable
    chip_info_t my_chip;

    // TODO: Initialize the chip with sample data
    init_chip(&my_chip, "STM32", "F407", 3.3);
    my_chip.temperature = 45.5;

    // TODO: Display chip information
    display_chip(&my_chip);

    // TODO: Check if chip is safe
    if (is_chip_safe(&my_chip)) {
        printf("Chip is operating safely\n");
    } else {
        printf("WARNING: Chip parameters out of range!\n");
    }

    return 0;
}
\end{lstlisting}

\textbf{Expected Output Example:}
\begin{verbatim}
=== Chip Information ===
Name: STM32
Part Number: F407
Voltage: 3.30V
Temperature: 45.50°C
Status: Working
Chip is operating safely
\end{verbatim}

\subsection{Task 3: Basic File Operations (60 minutes)}

\textbf{File:} \texttt{file\_operations.c}

\textbf{Requirements:}
\begin{itemize}
    \item Learn to read data from simple text files
    \item Practice writing data to files
    \item Handle basic file errors (file not found, etc.)
    \item Save and load chip information to/from files
\end{itemize}

\textbf{File Operations Template:}
\begin{lstlisting}[language=C]
#include <stdio.h>
#include <string.h>

// Simple chip structure (same as Task 2)
typedef struct {
    char name[20];
    char part_number[20];
    float voltage;
    float temperature;
    int is_working;
} chip_info_t;

// TODO: Create a function to save chip info to file
int save_chip_to_file(chip_info_t* chip, char filename[]) {
    // TODO: Open file for writing
    FILE* file = fopen(filename, "w");
    if (file == NULL) {
        printf("Error: Could not open file for writing\n");
        return 0;  // Failed
    }

    // TODO: Write chip information to file
    fprintf(file, "%s\n", chip->name);
    fprintf(file, "%s\n", chip->part_number);
    fprintf(file, "%.2f\n", chip->voltage);
    fprintf(file, "%.2f\n", chip->temperature);
    fprintf(file, "%d\n", chip->is_working);

    fclose(file);
    return 1;  // Success
}

// TODO: Create a function to load chip info from file
int load_chip_from_file(chip_info_t* chip, char filename[]) {
    // TODO: Open file for reading and load data
}

int main() {
    chip_info_t my_chip;

    // TODO: Create sample chip data
    strcpy(my_chip.name, "STM32");
    strcpy(my_chip.part_number, "F407");
    my_chip.voltage = 3.3;
    my_chip.temperature = 45.5;
    my_chip.is_working = 1;

    // TODO: Save chip to file
    if (save_chip_to_file(&my_chip, "chip_data.txt")) {
        printf("Chip data saved successfully!\n");
    }

    // TODO: Load chip from file and display
    chip_info_t loaded_chip;
    if (load_chip_from_file(&loaded_chip, "chip_data.txt")) {
        printf("Chip data loaded successfully!\n");
        // Display loaded chip info
    }

    return 0;
}
\end{lstlisting}

\subsection{Task 4: Simple Error Handling (45 minutes)}

\textbf{File:} \texttt{error\_handling.c}

\textbf{Requirements:}
\begin{itemize}
    \item Learn to check for and handle common errors
    \item Practice using return codes to indicate success/failure
    \item Add basic input validation to your programs
    \item Create helpful error messages for users
\end{itemize}

\textbf{Error Handling Examples:}
\begin{lstlisting}[language=C]
#include <stdio.h>
#include <string.h>

// Simple chip structure
typedef struct {
    char name[20];
    float voltage;
    float temperature;
} chip_info_t;

// TODO: Create a function that validates chip data
int validate_chip_data(chip_info_t* chip) {
    // TODO: Check if name is not empty
    if (strlen(chip->name) == 0) {
        printf("Error: Chip name cannot be empty\n");
        return 0;  // Invalid
    }

    // TODO: Check voltage range
    if (chip->voltage < 1.8 || chip->voltage > 3.6) {
        printf("Error: Voltage %.2fV is out of range (1.8V - 3.6V)\n",
               chip->voltage);
        return 0;  // Invalid
    }

    // TODO: Check temperature range
    if (chip->temperature < -40 || chip->temperature > 85) {
        printf("Error: Temperature %.1f°C is out of range (-40°C - 85°C)\n",
               chip->temperature);
        return 0;  // Invalid
    }

    return 1;  // Valid
}

// TODO: Create a function that safely gets user input
int get_chip_voltage(float* voltage) {
    printf("Enter chip voltage (1.8V - 3.6V): ");
    if (scanf("%f", voltage) != 1) {
        printf("Error: Please enter a valid number\n");
        return 0;  // Failed
    }
    return 1;  // Success
}

int main() {
    chip_info_t my_chip;

    // TODO: Get chip information with error checking
    printf("Enter chip name: ");
    scanf("%s", my_chip.name);

    if (!get_chip_voltage(&my_chip.voltage)) {
        printf("Failed to get voltage. Exiting.\n");
        return 1;
    }

    printf("Enter temperature: ");
    scanf("%f", &my_chip.temperature);

    // TODO: Validate all data before proceeding
    if (validate_chip_data(&my_chip)) {
        printf("All chip data is valid!\n");
        printf("Chip: %s, %.2fV, %.1f°C\n",
               my_chip.name, my_chip.voltage, my_chip.temperature);
    } else {
        printf("Please correct the errors and try again.\n");
    }

    return 0;
}
\end{lstlisting}

\subsection{Task 5: Comprehensive Chip Monitor (30 minutes)}

\textbf{File:} \texttt{chip\_monitor.c}

\textbf{Requirements:}
\begin{itemize}
    \item Combine all previous tasks into a simple monitoring program
    \item Use structures to organize chip information
    \item Add basic menu system for user interaction
    \item Practice integrating strings, structures, and file operations
\end{itemize}

\textbf{Simple Integration Example:}
\begin{lstlisting}[language=C]
#include <stdio.h>
#include <string.h>

// Use the chip structure from previous tasks
typedef struct {
    char name[20];
    char part_number[20];
    float voltage;
    float temperature;
    int is_working;
} chip_info_t;

// TODO: Create a simple menu function
void print_menu() {
    printf("\n=== Chip Monitor Menu ===\n");
    printf("1. Enter chip information\n");
    printf("2. Display chip information\n");
    printf("3. Save chip to file\n");
    printf("4. Load chip from file\n");
    printf("5. Exit\n");
    printf("Choose an option: ");
}

int main() {
    chip_info_t my_chip;
    int choice;
    char filename[50];

    printf("Welcome to Simple Chip Monitor!\n");

    do {
        print_menu();
        scanf("%d", &choice);

        switch (choice) {
            case 1:
                // TODO: Get chip information from user
                break;
            case 2:
                // TODO: Display chip information
                break;
            case 3:
                // TODO: Save chip to file
                break;
            case 4:
                // TODO: Load chip from file
                break;
            case 5:
                printf("Goodbye!\n");
                break;
            default:
                printf("Invalid choice. Please try again.\n");
        }
    } while (choice != 5);

    return 0;
}
\end{lstlisting}

\section{Part 2: Optional Extensions (For Advanced Students)}

\textbf{Note:} These tasks are completely optional and designed for students who finish early or want extra challenges. Focus on completing Part 1 first!

\subsection{Extension 1: Multiple Chip Management (Optional)}

\textbf{For students who want to work with arrays of structures:}
\begin{itemize}
    \item Create an array of chip structures
    \item Add functions to manage multiple chips
    \item Calculate statistics across all chips (average temperature, etc.)
\end{itemize}

\textbf{Simple Example:}
\begin{lstlisting}[language=C]
#include <stdio.h>

#define MAX_CHIPS 5

typedef struct {
    char name[20];
    float voltage;
    float temperature;
} chip_info_t;

int main() {
    chip_info_t chips[MAX_CHIPS];
    int chip_count = 0;

    // TODO: Add functions to manage multiple chips
    // add_chip(), display_all_chips(), find_hottest_chip()

    return 0;
}
\end{lstlisting}

\subsection{Extension 2: Enhanced File Operations (Optional)}

\textbf{For students who want to explore more file I/O:}
\begin{itemize}
    \item Save multiple chips to a single file
    \item Read chip data from CSV format
    \item Add backup and restore functionality
\end{itemize}

\subsection{Extension 3: Creative Enhancements (Optional)}

\textbf{Ideas for exploration:}
\begin{itemize}
    \item Add colorful output for different chip statuses
    \item Create ASCII art displays for chip information
    \item Add more chip parameters (frequency, power consumption)
    \item Implement a simple search function for chip names
\end{itemize}

\section{Submission Requirements}

\subsection{Code Quality Standards}
\begin{itemize}
    \item \textbf{String Safety:} Proper string handling with bounds checking
    \item \textbf{Structure Design:} Logical organization of data with appropriate types
    \item \textbf{File Operations:} Proper file handling with error checking
    \item \textbf{Error Handling:} Basic error checking and user-friendly messages
    \item \textbf{Code Style:} Clear variable names and consistent formatting
\end{itemize}

\subsection{Documentation Requirements}
\begin{itemize}
    \item \textbf{Function Comments:} Comment each function's purpose and parameters
    \item \textbf{README.md:} Updated with Day 3 functionality and usage instructions
    \item \textbf{Testing Notes:} Document what test cases you tried
    \item \textbf{Simple Comments:} Add comments explaining your logic
\end{itemize}

\section{Grading Rubric}

\begin{center}
\begin{tabular}{|l|c|l|}
\hline
\textbf{Component} & \textbf{Points} & \textbf{Criteria} \\
\hline
Task 1: String Handling & 25 & String functions, basic manipulation \\
Task 2: Simple Structures & 25 & Structure definition, initialization, usage \\
Task 3: File Operations & 25 & File I/O, error handling \\
Task 4: Error Handling & 20 & Input validation, error messages \\
Task 5: Integration & 15 & Menu system, combining concepts \\
Code Quality & 15 & Comments, style, variable naming \\
Documentation & 10 & README, clear explanations \\
GitHub Usage & 10 & Proper commits, submission \\
\hline
\textbf{Subtotal} & \textbf{145} & \textbf{Base assignment} \\
\hline
Optional Extensions & +15 & Extra credit for advanced features \\
\hline
\textbf{Total Possible} & \textbf{160} & \textbf{With extra credit} \\
\hline
\end{tabular}
\end{center}

\section{Common Challenges and Solutions}

\subsection{String Handling Issues}
\textbf{Problem:} String buffer overflows or incorrect string operations
\textbf{Solution:} Always check string lengths, use strncpy instead of strcpy when needed

\subsection{Structure Initialization}
\textbf{Problem:} Forgetting to initialize structure members
\textbf{Solution:} Always initialize all members when creating structures

\subsection{File Operation Errors}
\textbf{Problem:} Files not opening or data not saving correctly
\textbf{Solution:} Always check if file operations succeed, handle errors gracefully

\subsection{Input Validation}
\textbf{Problem:} Program crashes with invalid user input
\textbf{Solution:} Check scanf return values, validate input ranges

\section{Testing and Validation}

\subsection{Required Test Cases}
\begin{enumerate}
    \item \textbf{Pointer Tests:} NULL pointers, valid pointers, boundary conditions
    \item \textbf{Structure Tests:} Initialization, modification, validation
    \item \textbf{Bit Operation Tests:} All bit manipulation functions with various patterns
    \item \textbf{Memory Tests:} Allocation, deallocation, leak detection
    \item \textbf{Integration Tests:} Complete system functionality
\end{enumerate}

\subsection{AI Code Validation}
\begin{itemize}
    \item Compile and test all AI-generated code
    \item Verify AI code meets requirements
    \item Test edge cases not covered by AI
    \item Validate performance claims made by AI
    \item Ensure AI code follows course coding standards
\end{itemize}

\section{Extension Opportunities}

\subsection{Advanced Features}
\begin{itemize}
    \item \textbf{Custom Memory Allocator:} Implement pool-based allocation
    \item \textbf{Smart Pointers:} Reference counting for automatic cleanup
    \item \textbf{Bit Field Structures:} Use C bit fields for register modeling
    \item \textbf{Memory Mapping:} Simulate actual memory-mapped I/O
    \item \textbf{Endianness Handling:} Big-endian vs little-endian considerations
\end{itemize}

\subsection{AI Exploration}
\begin{itemize}
    \item Compare different AI tools (ChatGPT vs Copilot vs Claude)
    \item Experiment with different prompting strategies
    \item Use AI for code review and bug detection
    \item Explore AI-assisted refactoring techniques
\end{itemize}

\section{Success Tips}

\begin{itemize}
    \item \textbf{Start with Safety:} Always implement safety checks first
    \item \textbf{Test Incrementally:} Test each pointer operation as you implement it
    \item \textbf{Use AI Wisely:} Don't blindly accept AI suggestions, understand them first
    \item \textbf{Draw Memory Diagrams:} Visualize pointer relationships on paper
    \item \textbf{Validate Everything:} Check all inputs and intermediate results
    \item \textbf{Document AI Usage:} Keep detailed logs of AI interactions
\end{itemize}

\vspace{1cm}

\begin{center}
\textbf{Master the power of pointers and structures!}\\
\textit{These are the building blocks of embedded systems programming.}
\end{center}

\end{document}

