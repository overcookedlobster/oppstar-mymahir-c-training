\documentclass[11pt,a4paper]{article}
\usepackage[utf8]{inputenc}
\usepackage[T1]{fontenc}
\usepackage{geometry}
\usepackage{graphicx}
\usepackage{xcolor}
\usepackage{listings}
\usepackage{fancyhdr}
\usepackage{titlesec}
\usepackage{hyperref}
\usepackage{enumitem}
\usepackage{booktabs}
\usepackage{array}

% Page setup
\geometry{margin=1in}
\setlength{\headheight}{14pt}
\pagestyle{fancy}
\fancyhf{}
\fancyhead[L]{Day 6 Assignment: Capstone Project}
\fancyhead[R]{\thepage}
\fancyfoot[C]{C Programming for Post-Silicon Validation Engineers}

% Colors
\definecolor{codeblue}{RGB}{0,102,204}
\definecolor{codegray}{RGB}{128,128,128}
\definecolor{codegreen}{RGB}{0,128,0}
\definecolor{backcolour}{RGB}{245,245,245}

% Code listing style
\lstdefinestyle{cstyle}{
    backgroundcolor=\color{backcolour},
    commentstyle=\color{codegreen},
    keywordstyle=\color{codeblue},
    numberstyle=\tiny\color{codegray},
    stringstyle=\color{red},
    basicstyle=\ttfamily\footnotesize,
    breakatwhitespace=false,
    breaklines=true,
    captionpos=b,
    keepspaces=true,
    numbers=left,
    numbersep=5pt,
    showspaces=false,
    showstringspaces=false,
    showtabs=false,
    tabsize=2,
    frame=single
}

\lstset{style=cstyle}

% Title formatting
\titleformat{\section}{\Large\bfseries\color{codeblue}}{\thesection}{1em}{}
\titleformat{\subsection}{\large\bfseries}{\thesubsection}{1em}{}

\hypersetup{
    colorlinks=true,
    linkcolor=codeblue,
    filecolor=magenta,
    urlcolor=cyan,
    pdftitle={Day 6 Assignment - Capstone Project},
    pdfauthor={Yahwista Salomo},
}

\begin{document}

% Header
\begin{center}
    {\Huge\bfseries\color{codeblue} DAY 6 CAPSTONE PROJECT}\\[0.5cm]
    {\Large Comprehensive Validation System Integration}\\[0.3cm]
    {\large Team-Based Final Project and Portfolio Development}\\[0.2cm]
    {\normalsize Duration: 6 Hours + Presentation}
\end{center}

\vspace{1cm}

\section{Project Overview}

\subsection{Mission Statement}
Design, implement, and demonstrate a comprehensive validation system that integrates all course concepts into a professional-grade solution suitable for post-silicon validation engineering roles. This capstone project serves as the culmination of your learning journey and the foundation of your professional portfolio.

\subsection{Learning Objectives}
By completing this capstone project, you will:
\begin{itemize}
    \item Integrate all course concepts into a cohesive, working system
    \item Demonstrate professional project management and collaboration skills
    \item Create portfolio-quality documentation and presentations
    \item Apply AI tools effectively while maintaining critical evaluation
    \item Build a foundation for continued learning and career development
    \item Establish professional connections within your cohort
\end{itemize}

\subsection{Project Scope}
Your team will create a complete validation system that demonstrates mastery of:
\begin{itemize}
    \item C programming fundamentals and advanced concepts
    \item Embedded systems programming for MicroBlaze-V
    \item Hardware-in-the-loop testing methodologies
    \item Professional software development practices
    \item Technical communication and documentation
\end{itemize}

\section{Team Formation and Roles}

\subsection{Team Structure}
\begin{itemize}
    \item \textbf{Team Size:} 2-3 participants per team
    \item \textbf{Formation:} Self-selected or instructor-assigned based on complementary skills
    \item \textbf{Diversity:} Mix of backgrounds, experience levels, and project interests encouraged
\end{itemize}

\subsection{Recommended Roles}
\textbf{For 3-Person Teams:}
\begin{itemize}
    \item \textbf{Project Lead:} Overall coordination, integration, and timeline management
    \item \textbf{Hardware Specialist:} MicroBlaze-V programming, peripheral integration, debugging
    \item \textbf{Software Architect:} Code structure, optimization, testing framework
\end{itemize}

\textbf{For 2-Person Teams:}
\begin{itemize}
    \item \textbf{Lead Developer:} Project coordination and core implementation
    \item \textbf{Validation Engineer:} Testing, documentation, and quality assurance
\end{itemize}

\textbf{Shared Responsibilities:}
\begin{itemize}
    \item Code review and quality assurance
    \item Documentation and presentation preparation
    \item AI tool usage and evaluation
    \item GitHub workflow and version control
\end{itemize}

\section{Project Requirements}

\subsection{Core Requirements (Must Have - 70\% of Grade)}

\subsubsection{1. Multi-Peripheral Validation System}
\begin{itemize}
    \item Test at least 3 different MicroBlaze-V peripherals (GPIO, ADC, timers, etc.)
    \item Implement comprehensive test suites for each peripheral
    \item Provide pass/fail criteria and detailed reporting
    \item Include error detection and recovery mechanisms
\end{itemize}

\subsubsection{2. Modular Software Architecture}
\begin{itemize}
    \item Clean separation of concerns with header files
    \item Hardware abstraction layer for portability
    \item Reusable validation library components
    \item Professional code organization and documentation
\end{itemize}

\subsubsection{3. Cross-Platform Compatibility}
\begin{itemize}
    \item Code compiles and runs on both desktop and MicroBlaze-V
    \item CMake build system supporting both targets
    \item Consistent behavior across platforms (where applicable)
    \item Platform-specific optimizations where needed
\end{itemize}

\subsubsection{4. Hardware-in-the-Loop Testing}
\begin{itemize}
    \item Actual hardware testing on MicroBlaze-V platform
    \item Real-time constraint validation
    \item Hardware-specific validation scenarios
    \item Integration with physical test setup
\end{itemize}

\subsubsection{5. Professional Documentation}
\begin{itemize}
    \item Comprehensive README with setup and usage instructions
    \item API documentation for all public functions
    \item Architecture overview and design decisions
    \item Test results and validation reports
\end{itemize}

\subsection{Advanced Features (Choose 2+ - 20\% of Grade)}

\subsubsection{A. Fault Injection and Recovery Testing}
\begin{itemize}
    \item Controlled fault injection mechanisms
    \item System recovery validation
    \item Resilience testing under stress conditions
    \item Graceful degradation strategies
\end{itemize}

\subsubsection{B. Statistical Analysis and Reporting}
\begin{itemize}
    \item Test result statistical analysis
    \item Performance benchmarking and trends
    \item Automated report generation
    \item Data visualization (ASCII charts or export to CSV)
\end{itemize}

\subsubsection{C. Real-Time Performance Monitoring}
\begin{itemize}
    \item Execution time measurement and analysis
    \item Memory usage monitoring
    \item CPU utilization tracking
    \item Performance optimization recommendations
\end{itemize}

\subsubsection{D. Advanced Communication Protocols}
\begin{itemize}
    \item UART, SPI, or I2C validation testing
    \item Protocol compliance verification
    \item Communication error detection and handling
    \item Multi-device communication scenarios
\end{itemize}

\subsubsection{E. AI-Enhanced Optimization}
\begin{itemize}
    \item AI-assisted code optimization
    \item Automated test case generation using AI
    \item AI-powered anomaly detection in test results
    \item Critical evaluation and improvement of AI suggestions
\end{itemize}

\subsubsection{F. Custom Validation Protocols}
\begin{itemize}
    \item Industry-specific validation scenarios
    \item Custom test methodologies
    \item Innovative validation approaches
    \item Integration with external test equipment (simulated)
\end{itemize}

\subsection{Innovation and Creativity (10\% of Grade)}
\begin{itemize}
    \item Novel approaches to validation challenges
    \item Creative use of MicroBlaze-V capabilities
    \item Innovative user interfaces or interaction methods
    \item Unique integration of course concepts
    \item Original solutions to technical problems
\end{itemize}

\section{Project Timeline and Milestones}

\subsection{Hour-by-Hour Schedule}

\textbf{Hour 1: Project Planning and Setup (9:00-10:00 AM)}
\begin{itemize}
    \item Team formation and role assignment
    \item Project theme selection and scope definition
    \item GitHub repository setup and branch strategy
    \item Architecture design and module breakdown
    \item Task assignment and timeline planning
\end{itemize}

\textbf{Hour 2: Core Implementation Phase 1 (10:15-11:15 AM)}
\begin{itemize}
    \item Basic project structure creation
    \item Core validation framework implementation
    \item Hardware abstraction layer development
    \item Initial peripheral testing modules
\end{itemize}

\textbf{Hour 3: Core Implementation Phase 2 (11:30 AM-12:30 PM)}
\begin{itemize}
    \item Multi-peripheral integration
    \item Cross-compilation setup and testing
    \item Hardware testing on MicroBlaze-V
    \item Error handling and recovery implementation
\end{itemize}

\textbf{Hour 4: Advanced Features and Integration (1:30-2:30 PM)}
\begin{itemize}
    \item Advanced feature implementation
    \item System integration and testing
    \item Performance optimization
    \item AI tool integration and evaluation
\end{itemize}

\textbf{Hour 5: Testing, Documentation, and Polish (2:45-3:45 PM)}
\begin{itemize}
    \item Comprehensive system testing
    \item Documentation completion
    \item Code review and quality assurance
    \item Presentation preparation
\end{itemize}

\textbf{Hour 6: Final Integration and Presentation Prep (4:00-5:00 PM)}
\begin{itemize}
    \item Final testing and bug fixes
    \item Presentation rehearsal
    \item Demo preparation and hardware setup
    \item Portfolio documentation finalization
\end{itemize}

\subsection{Milestone Checkpoints}
\textbf{Every hour, teams will demonstrate:}
\begin{itemize}
    \item Progress toward milestone goals
    \item Working code compilation and basic functionality
    \item Updated GitHub repository with commits
    \item Any blockers or assistance needed
    \item Adjustment of goals if necessary
\end{itemize}

\section{Suggested Project Themes}

\subsection{Theme 1: Multi-Chip Validation Network}
\textbf{Concept:} Create a system that can validate multiple MicroBlaze-V boards simultaneously
\begin{itemize}
    \item Master-slave communication architecture
    \item Distributed testing coordination
    \item Comparative analysis between chips
    \item Network fault tolerance and recovery
\end{itemize}

\subsection{Theme 2: Environmental Stress Testing System}
\textbf{Concept:} Validate chip behavior under simulated environmental conditions
\begin{itemize}
    \item Temperature variation simulation
    \item Voltage fluctuation testing
    \item Timing stress under load
    \item Environmental data logging and analysis
\end{itemize}

\subsection{Theme 3: Communication Protocol Validator}
\textbf{Concept:} Comprehensive testing of communication interfaces
\begin{itemize}
    \item UART, SPI, I2C protocol validation
    \item Error injection and recovery testing
    \item Protocol compliance verification
    \item Multi-protocol coordination testing
\end{itemize}

\subsection{Theme 4: Power Management Validation Suite}
\textbf{Concept:} Test and validate power management features
\begin{itemize}
    \item Sleep mode validation
    \item Power consumption measurement
    \item Wake-up source testing
    \item Power efficiency optimization
\end{itemize}

\subsection{Theme 5: Signal Integrity and Timing Analyzer}
\textbf{Concept:} Validate signal quality and timing characteristics
\begin{itemize}
    \item Signal rise/fall time measurement
    \item Clock jitter analysis
    \item Setup and hold time validation
    \item Signal integrity under different loads
\end{itemize}

\subsection{Theme 6: Automated Regression Testing Framework}
\textbf{Concept:} Continuous validation and regression testing system
\begin{itemize}
    \item Automated test scheduling
    \item Regression detection algorithms
    \item Historical trend analysis
    \item Continuous integration simulation
\end{itemize}

\section{Technical Implementation Guidelines}

\subsection{Code Architecture Requirements}

\textbf{Directory Structure:}
\begin{verbatim}
capstone_project/
|-- src/
|   |-- main.c
|   |-- validation_core.c
|   |-- [peripheral]_validator.c
|   |-- hardware_abstraction.c
|   |-- test_framework.c
|   +-- [advanced_feature].c
|-- include/
|   |-- validation_core.h
|   |-- [peripheral]_validator.h
|   |-- hardware_abstraction.h
|   |-- test_framework.h
|   |-- common_types.h
|   +-- [advanced_feature].h
|-- tests/
|   |-- unit_tests.c
|   +-- integration_tests.c
|-- docs/
|   |-- README.md
|   |-- API_REFERENCE.md
|   |-- ARCHITECTURE.md
|   |-- USER_GUIDE.md
|   +-- TEST_RESULTS.md
|-- scripts/
|   |-- build_desktop.sh
|   |-- build_pico.sh
|   +-- run_tests.sh
|-- CMakeLists.txt
+-- .github/
    +-- workflows/
        +-- ci.yml
\end{verbatim}

\subsection{Code Quality Standards}
\begin{itemize}
    \item \textbf{Compilation:} Zero warnings with \texttt{-Wall -Wextra}
    \item \textbf{Documentation:} All public functions documented
    \item \textbf{Error Handling:} Comprehensive error checking and recovery
    \item \textbf{Memory Safety:} No memory leaks or unsafe operations
    \item \textbf{Performance:} Optimized for embedded constraints
    \item \textbf{Portability:} Works on both desktop and embedded platforms
\end{itemize}

\subsection{AI Integration Requirements}
\begin{itemize}
    \item Document all AI assistance in \texttt{AI\_USAGE\_LOG.md}
    \item Provide critical evaluation of AI suggestions
    \item Show improvements made to AI-generated code
    \item Demonstrate understanding of all AI-assisted implementations
    \item Include AI evaluation in final presentation
\end{itemize}

\section{Presentation Requirements}

\subsection{Presentation Format}
\begin{itemize}
    \item \textbf{Duration:} 7-10 minutes per team
    \item \textbf{Format:} Live demonstration with slides
    \item \textbf{Audience:} Classmates, instructors, and optional industry guests
    \item \textbf{Equipment:} Projector, MicroBlaze-V hardware, laptop
\end{itemize}

\subsection{Presentation Structure}

\textbf{1. Introduction (1 minute)}
\begin{itemize}
    \item Team member introductions and roles
    \item Project theme and motivation
    \item High-level overview of solution
\end{itemize}

\textbf{2. Problem Statement (1 minute)}
\begin{itemize}
    \item Validation challenge being addressed
    \item Why this problem is important in post-silicon validation
    \item Success criteria and goals
\end{itemize}

\textbf{3. Solution Architecture (2 minutes)}
\begin{itemize}
    \item High-level system design
    \item Key components and their interactions
    \item Technology choices and rationale
    \item Integration of course concepts
\end{itemize}

\textbf{4. Technical Implementation (3 minutes)}
\begin{itemize}
    \item Key algorithms and data structures
    \item Interesting technical challenges and solutions
    \item Code quality and architecture highlights
    \item AI integration and evaluation
\end{itemize}

\textbf{5. Live Demonstration (2 minutes)}
\begin{itemize}
    \item Hardware setup and system initialization
    \item Core functionality demonstration
    \item Advanced features showcase
    \item Error handling and recovery demonstration
\end{itemize}

\textbf{6. Results and Lessons Learned (1 minute)}
\begin{itemize}
    \item Key achievements and metrics
    \item Challenges overcome and lessons learned
    \item Future improvements and extensions
    \item Relevance to validation engineering careers
\end{itemize}

\subsection{Presentation Evaluation Criteria}
\begin{itemize}
    \item \textbf{Technical Content:} Accuracy and depth of technical explanation
    \item \textbf{Demonstration:} Successful live demo of working system
    \item \textbf{Communication:} Clear, professional presentation style
    \item \textbf{Team Coordination:} Effective collaboration and role distribution
    \item \textbf{Innovation:} Creative approaches and novel solutions
\end{itemize}

\section{Grading Rubric}

\begin{center}
\begin{tabular}{|l|c|l|}
\hline
\textbf{Component} & \textbf{Points} & \textbf{Criteria} \\
\hline
\multicolumn{3}{|c|}{\textbf{Technical Implementation (60\%)}} \\
\hline
Core Requirements & 40 & Multi-peripheral validation, architecture, cross-platform \\
Advanced Features & 15 & Implementation of 2+ advanced features \\
Code Quality & 15 & Style, documentation, error handling, performance \\
\hline
\multicolumn{3}{|c|}{\textbf{Collaboration and Process (25\%)}} \\
\hline
GitHub Workflow & 10 & Commit quality, branching, pull requests \\
Team Collaboration & 10 & Effective teamwork, role distribution \\
Project Management & 5 & Timeline adherence, milestone completion \\
\hline
\multicolumn{3}{|c|}{\textbf{Communication and Innovation (15\%)}} \\
\hline
Presentation & 8 & Technical accuracy, demonstration, communication \\
Documentation & 4 & README, API docs, architecture documentation \\
Innovation & 3 & Creative solutions, novel approaches \\
\hline
\textbf{Total} & \textbf{100} & \textbf{Professional validation engineering project} \\
\hline
\end{tabular}
\end{center}

\subsection{Grading Scale}
\begin{itemize}
    \item \textbf{A (90-100\%):} Exceptional project demonstrating mastery of all concepts
    \item \textbf{B (80-89\%):} Strong project meeting all core requirements with good execution
    \item \textbf{C (70-79\%):} Satisfactory project meeting basic requirements
    \item \textbf{D (60-69\%):} Below expectations with significant issues or missing components
    \item \textbf{F (<60\%):} Unsatisfactory project not meeting minimum requirements
\end{itemize}

\section{Support and Resources}

\subsection{Available Support}
\begin{itemize}
    \item \textbf{Instructors:} Available throughout the day for guidance and troubleshooting
    \item \textbf{Teaching Assistants:} Hands-on technical support and debugging help
    \item \textbf{Peer Teams:} Cross-team collaboration and knowledge sharing encouraged
    \item \textbf{AI Tools:} Access to coding assistants with evaluation requirements
    \item \textbf{Hardware:} Backup MicroBlaze-V boards and debugging equipment
\end{itemize}

\subsection{Emergency Procedures}
\begin{itemize}
    \item \textbf{Hardware Failure:} Backup boards available, simulation fallback options
    \item \textbf{Build Issues:} TA support for toolchain and configuration problems
    \item \textbf{Git Conflicts:} Instructor assistance with version control issues
    \item \textbf{Team Conflicts:} Mediation and role adjustment support
    \item \textbf{Scope Issues:} Guidance on scope adjustment and priority focus
\end{itemize}

\subsection{Resource Library}
\begin{itemize}
    \item All previous day's code examples and solutions
    \item MicroBlaze-V datasheet and SDK documentation
    \item Course slides and reference materials
    \item Example project structures and templates
    \item Debugging guides and troubleshooting resources
\end{itemize}

\section{Portfolio Development}

\subsection{GitHub Portfolio Requirements}
Your capstone project will serve as a centerpiece of your professional portfolio. Ensure your GitHub repository includes:

\begin{itemize}
    \item \textbf{Professional README:} Clear project description, setup instructions, and usage examples
    \item \textbf{Code Quality:} Well-organized, commented, and documented source code
    \item \textbf{Documentation:} Comprehensive technical documentation and user guides
    \item \textbf{Demonstration Materials:} Videos, screenshots, or GIFs showing the system in action
    \item \textbf{Technical Analysis:} Design decisions, trade-offs, and performance analysis
    \item \textbf{Professional Profile:} Updated GitHub profile highlighting your new skills
\end{itemize}

\subsection{Career Integration}
\begin{itemize}
    \item Link your project to LinkedIn profile and resume
    \item Highlight validation engineering skills and embedded systems experience
    \item Showcase collaborative development and professional practices
    \item Demonstrate continuous learning and AI tool proficiency
    \item Build connections with classmates for professional networking
\end{itemize}

\section{Post-Project Activities}

\subsection{Immediate Follow-Up (Week 1)}
\begin{itemize}
    \item Complete any remaining documentation
    \item Reflect on lessons learned and areas for improvement
    \item Connect with team members and classmates on LinkedIn
    \item Update resume and portfolio with new skills and projects
    \item Begin planning for extended homework assignments
\end{itemize}

\subsection{Extended Development (Weeks 2-4)}
\begin{itemize}
    \item Implement additional advanced features
    \item Optimize performance and add new capabilities
    \item Create detailed technical blog posts about your project
    \item Contribute to open-source embedded systems projects
    \item Mentor future course participants
\end{itemize}

\section{Success Tips and Best Practices}

\subsection{Project Management}
\begin{itemize}
    \item \textbf{Start with MVP:} Get basic functionality working first, then add features
    \item \textbf{Communicate Frequently:} Regular team check-ins and status updates
    \item \textbf{Version Control:} Commit early and often with descriptive messages
    \item \textbf{Test Continuously:} Don't wait until the end to test integration
    \item \textbf{Document as You Go:} Write documentation while implementation is fresh
\end{itemize}

\subsection{Technical Development}
\begin{itemize}
    \item \textbf{Modular Design:} Build reusable components that can be easily tested
    \item \textbf{Error Handling:} Plan for failure modes and recovery strategies
    \item \textbf{Performance Focus:} Keep embedded constraints in mind throughout development
    \item \textbf{Code Reviews:} Review each other's code for quality and learning
    \item \textbf{Hardware Testing:} Test on actual hardware early and often
\end{itemize}

\subsection{Presentation Preparation}
\begin{itemize}
    \item \textbf{Practice Demo:} Rehearse your demonstration multiple times
    \item \textbf{Backup Plans:} Have fallback options if hardware fails during demo
    \item \textbf{Tell a Story:} Structure your presentation as a compelling narrative
    \item \textbf{Time Management:} Practice staying within the time limit
    \item \textbf{Engage Audience:} Make eye contact and explain technical concepts clearly
\end{itemize}

\section{Reflection and Evaluation}

\subsection{Self-Assessment Questions}
At the end of the project, reflect on:
\begin{itemize}
    \item What was your biggest technical achievement in this project?
    \item How did you overcome the most challenging obstacle you faced?
    \item What would you do differently if you started the project over?
    \item How will you apply these skills in your future career?
    \item What additional skills do you want to develop next?
\end{itemize}

\subsection{Peer Evaluation}
Provide feedback on your teammates':
\begin{itemize}
    \item Technical contributions and problem-solving skills
    \item Communication and collaboration effectiveness
    \item Reliability and commitment to the project
    \item Leadership and initiative in driving progress
    \item Areas for improvement and growth
\end{itemize}

\vspace{1cm}

\begin{center}
\textbf{Congratulations on reaching the capstone!}\\
\textit{This project represents the culmination of your transformation into a validation engineer.}\\
\textbf{Show the world what you've learned and built!}
\end{center}

\end{document}

