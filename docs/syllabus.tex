\documentclass[11pt,a4paper]{article}
\usepackage[utf8]{inputenc}
\usepackage[T1]{fontenc}
\usepackage{geometry}
\usepackage{graphicx}
\usepackage{xcolor}
\usepackage{fancyhdr}
\usepackage{titlesec}
\usepackage{hyperref}
\usepackage{enumitem}
\usepackage{booktabs}
\usepackage{array}
\usepackage{longtable}
\usepackage{multirow}

% Page setup
\geometry{margin=1in}
\pagestyle{fancy}
\fancyhf{}
\fancyhead[L]{C Programming for Post-Silicon Validation - Syllabus}
\fancyhead[R]{\thepage}
\fancyfoot[C]{6-Day Intensive Course}

% Colors
\definecolor{courseblue}{RGB}{0,102,204}
\definecolor{sectiongray}{RGB}{240,240,240}

% Title formatting
\titleformat{\section}{\Large\bfseries\color{courseblue}}{\thesection}{1em}{}
\titleformat{\subsection}{\large\bfseries}{\thesubsection}{1em}{}

\hypersetup{
    colorlinks=true,
    linkcolor=courseblue,
    filecolor=magenta,
    urlcolor=cyan,
    pdftitle={C Programming for Post-Silicon Validation - Syllabus},
    pdfauthor={Course Instructor},
}

\begin{document}

% Header
\begin{center}
    {\Huge\bfseries\color{courseblue} COURSE SYLLABUS}\\[0.5cm]
    {\Large C Programming for Post-Silicon Validation Engineers}\\[0.3cm]
    {\large 6-Day Intensive Bootcamp}\\[0.2cm]
    {\normalsize Version 1.0 - 2025}
\end{center}

\vspace{1cm}

\section{Course Information}

\begin{tabular}{ll}
\textbf{Course Title:} & C Programming for Post-Silicon Validation Engineers \\
\textbf{Duration:} & 6 Days (42 Contact Hours) \\
\textbf{Format:} & Intensive Bootcamp (In-Person/Hybrid) \\
\textbf{Class Size:} & 20 Participants Maximum \\
\textbf{Prerequisites:} & None (Designed for Programming Beginners) \\
\textbf{Target Audience:} & Engineers Transitioning to Post-Silicon Validation \\
\textbf{Hardware Platform:} & Raspberry Pi Pico (RP2040) \\
\textbf{Primary Tools:} & GCC, GDB, GitHub, VS Code, Pico SDK \\
\end{tabular}

\section{Instructor Information}

\begin{tabular}{ll}
\textbf{Lead Instructor:} & [Instructor Name] \\
\textbf{Email:} & [instructor@company.com] \\
\textbf{Office Hours:} & By appointment during course week \\
\textbf{Teaching Assistants:} & 1-2 TAs for hands-on support \\
\textbf{Response Time:} & Within 4 hours during course days \\
\end{tabular}

\section{Course Description}

This intensive 6-day course transforms engineers with no programming experience into proficient C developers capable of creating embedded validation tools for semiconductor post-silicon testing. The curriculum emphasizes practical, hands-on learning with real hardware and industry-standard development practices.

Participants will progress from basic C syntax to developing complete validation systems running on ARM Cortex-M0+ microcontrollers. All examples and projects simulate real-world post-silicon validation scenarios, including register manipulation, fault detection, timing analysis, and automated testing procedures.

\section{Learning Objectives}

Upon successful completion of this course, participants will be able to:

\subsection{Technical Skills}
\begin{enumerate}
    \item Write, compile, and debug C programs using industry-standard tools (GCC, GDB)
    \item Cross-compile applications for ARM Cortex-M0+ embedded targets
    \item Develop embedded applications using the Raspberry Pi Pico SDK
    \item Implement bit manipulation and register access patterns common in validation
    \item Create modular, maintainable code following industry best practices
    \item Use version control (Git/GitHub) for collaborative development
    \item Apply debugging techniques for both native and embedded targets
\end{enumerate}

\subsection{Validation-Specific Skills}
\begin{enumerate}
    \item Design and implement automated test sequences for hardware validation
    \item Simulate fault injection and detection mechanisms
    \item Create register monitoring and manipulation routines
    \item Develop timing-critical validation procedures
    \item Implement data logging and result reporting systems
    \item Build hardware-in-the-loop testing frameworks
\end{enumerate}

\subsection{Professional Skills}
\begin{enumerate}
    \item Collaborate effectively using GitHub workflows
    \item Document code and projects professionally
    \item Present technical work to peers and stakeholders
    \item Evaluate and integrate AI coding assistance responsibly
    \item Plan and execute complex technical projects
    \item Build a professional portfolio of validation projects
\end{enumerate}

\section{Course Schedule}

\subsection{Daily Structure}
Each day follows a consistent 7-hour format:
\begin{itemize}
    \item \textbf{Morning Session (2.5 hours):} Interactive lectures with live coding demonstrations
    \item \textbf{Break (30 minutes):} Networking and informal discussion
    \item \textbf{Afternoon Session (3.5 hours):} Hands-on labs and collaborative projects
    \item \textbf{Wrap-up (30 minutes):} Q\&A, troubleshooting, and next-day preview
    \item \textbf{Evening (1-2 hours):} Optional homework and reading assignments
\end{itemize}

\subsection{Weekly Overview}

\begin{longtable}{|p{1.5cm}|p{4cm}|p{6cm}|p{3cm}|}
\hline
\textbf{Day} & \textbf{Focus Area} & \textbf{Key Topics} & \textbf{Deliverables} \\
\hline
\endhead

\textbf{Day 1} & C Fundamentals \& Compilation &
\begin{itemize}[nosep]
    \item Variables and data types
    \item Operators and expressions
    \item Input/output operations
    \item GCC compilation process
    \item Basic validation calculations
\end{itemize} &
\begin{itemize}[nosep]
    \item Voltage validator program
    \item GitHub repository setup
    \item Compilation workflow
\end{itemize} \\
\hline

\textbf{Day 2} & Control Flow \& Debugging &
\begin{itemize}[nosep]
    \item Conditional statements
    \item Loops and iteration
    \item Function definition/calling
    \item GDB debugging basics
    \item Test sequence design
\end{itemize} &
\begin{itemize}[nosep]
    \item Register monitor program
    \item Debugging session logs
    \item Function library
\end{itemize} \\
\hline

\textbf{Day 3} & Memory \& Data Structures &
\begin{itemize}[nosep]
    \item Pointers and memory addressing
    \item Arrays and strings
    \item Structures and typedef
    \item Bit manipulation operations
    \item AI tool introduction
\end{itemize} &
\begin{itemize}[nosep]
    \item Chip state monitor
    \item Bit manipulation library
    \item AI evaluation report
\end{itemize} \\
\hline

\textbf{Day 4} & Advanced Functions \& Cross-Compilation &
\begin{itemize}[nosep]
    \item Modular programming
    \item Header files and linking
    \item Cross-compilation setup
    \item Pico SDK introduction
    \item CMake build system
\end{itemize} &
\begin{itemize}[nosep]
    \item Modular test suite
    \item ARM cross-compiled binary
    \item CMake configuration
\end{itemize} \\
\hline

\textbf{Day 5} & Hardware Debugging \& RP2040 &
\begin{itemize}[nosep]
    \item Advanced GDB techniques
    \item RP2040 GPIO control
    \item Peripheral programming
    \item Hardware-in-the-loop testing
    \item Fault injection methods
\end{itemize} &
\begin{itemize}[nosep]
    \item GPIO validation suite
    \item Hardware test results
    \item Debugging documentation
\end{itemize} \\
\hline

\textbf{Day 6} & Capstone Project \& Integration &
\begin{itemize}[nosep]
    \item Project planning and design
    \item Full system integration
    \item Results presentation
    \item Portfolio development
    \item Career planning
\end{itemize} &
\begin{itemize}[nosep]
    \item Complete validation system
    \item Project presentation
    \item Professional portfolio
\end{itemize} \\
\hline

\end{longtable}

\section{Assessment Methods}

\subsection{Grading Distribution}
\begin{center}
\begin{tabular}{|l|c|l|}
\hline
\textbf{Component} & \textbf{Weight} & \textbf{Description} \\
\hline
Daily Lab Assignments & 35\% & Auto-graded GitHub submissions \\
Capstone Project & 35\% & Comprehensive validation system \\
Participation \& Collaboration & 30\% & GitHub activity, peer feedback \\
\hline
\textbf{Total} & \textbf{100\%} & \\
\hline
\end{tabular}
\end{center}

\subsection{Assessment Criteria}

\subsubsection{Daily Labs (35\%)}
\begin{itemize}
    \item Code compilation and execution (auto-graded)
    \item Unit test passage rates
    \item Code quality and style adherence
    \item Documentation completeness
    \item GitHub workflow compliance
\end{itemize}

\subsubsection{Capstone Project (35\%)}
\begin{itemize}
    \item Technical implementation quality (40\%)
    \item Problem-solving approach (25\%)
    \item Documentation and presentation (20\%)
    \item Innovation and creativity (15\%)
\end{itemize}

\subsubsection{Participation \& Collaboration (30\%)}
\begin{itemize}
    \item Active engagement in discussions and activities
    \item Quality of peer code reviews and feedback
    \item Contribution to team projects
    \item Help provided to fellow participants
    \item Professional communication in GitHub
\end{itemize}

\subsection{Success Metrics}
\begin{itemize}
    \item \textbf{Proficiency Threshold:} 80\% overall score required for certification
    \item \textbf{Skill Demonstration:} Successful completion of all core lab assignments
    \item \textbf{Portfolio Quality:} Professional-grade GitHub repository with documented projects
    \item \textbf{Peer Evaluation:} Positive feedback from collaborative activities
\end{itemize}

\section{Required Materials}

\subsection{Hardware (Provided)}
\begin{itemize}
    \item Raspberry Pi Pico (RP2040 microcontroller board)
    \item USB-A to micro-USB cable
    \item Breadboard and jumper wires (for advanced projects)
    \item LEDs and resistors for visual feedback
    \item Backup hardware for troubleshooting
\end{itemize}

\subsection{Software (Free/Open Source)}
\begin{itemize}
    \item GCC compiler suite (native and ARM cross-compiler)
    \item Git version control system
    \item VS Code editor with C/C++ extensions
    \item Raspberry Pi Pico SDK
    \item CMake build system
    \item GDB debugger
    \item QEMU emulator (backup/simulation)
\end{itemize}

\subsection{Accounts and Access}
\begin{itemize}
    \item GitHub account (free tier sufficient)
    \item Access to course GitHub Classroom organization
    \item Course Slack workspace (optional for communication)
    \item Shared Google Drive for resources (optional)
\end{itemize}

\section{Pre-Course Requirements}

\subsection{Mandatory Setup (Due 1 Week Before Course)}
\begin{enumerate}
    \item Complete development environment installation following provided guide
    \item Verify tool functionality with provided test programs
    \item Complete Git/GitHub tutorial (2-3 hours)
    \item Watch introductory C programming videos (CS50 lectures 1-2)
    \item Submit setup confirmation via GitHub issue
\end{enumerate}

\subsection{Recommended Preparation}
\begin{itemize}
    \item Review basic digital electronics concepts
    \item Familiarize yourself with command-line interfaces
    \item Read overview of post-silicon validation processes
    \item Join course communication channels
\end{itemize}

\section{Course Policies}

\subsection{Attendance Policy}
\begin{itemize}
    \item Full attendance required for all 6 days
    \item Maximum 2 hours of excused absence with prior notification
    \item Make-up sessions available for emergency absences
    \item Late arrivals must catch up during breaks or after hours
\end{itemize}

\subsection{Collaboration Policy}
\begin{itemize}
    \item \textbf{Encouraged:} Pair programming, code reviews, debugging assistance
    \item \textbf{Required:} Attribution of help received in commit messages
    \item \textbf{Prohibited:} Direct copying of solutions without understanding
    \item \textbf{AI Tools:} Allowed from Day 3 onward with documentation requirements
\end{itemize}

\subsection{Late Submission Policy}
\begin{itemize}
    \item Daily labs: 10\% penalty per day late (maximum 3 days)
    \item Capstone project: No late submissions accepted
    \item Extensions available for documented emergencies
    \item GitHub timestamps used for submission verification
\end{itemize}

\subsection{Academic Integrity}
\begin{itemize}
    \item All submitted work must be your own or properly attributed
    \item Collaboration must be documented in commit messages
    \item AI assistance must be acknowledged and evaluated
    \item Plagiarism detection tools used on all submissions
    \item Violations result in course failure and reporting
\end{itemize}

\section{Support Resources}

\subsection{During Course Hours}
\begin{itemize}
    \item Lead instructor for conceptual questions and advanced topics
    \item Teaching assistants for hands-on debugging and technical support
    \item Peer support through structured pair programming
    \item Real-time help via course communication channels
\end{itemize}

\subsection{Outside Course Hours}
\begin{itemize}
    \item Recorded lecture materials for review
    \item Comprehensive documentation and reference materials
    \item Online forums for asynchronous Q\&A
    \item Office hours by appointment
    \item Emergency contact for critical technical issues
\end{itemize}

\subsection{Technical Support}
\begin{itemize}
    \item Hardware troubleshooting and replacement
    \item Software installation and configuration assistance
    \item GitHub and development workflow support
    \item Backup development environments (Docker containers)
    \item Alternative solutions for accessibility needs
\end{itemize}

\section{Post-Course Development}

\subsection{Extended Learning Program (Optional)}
\begin{itemize}
    \item 4-week structured homework assignments
    \item Weekly virtual check-ins with instructors
    \item Advanced project challenges
    \item Portfolio development guidance
    \item Career transition support
\end{itemize}

\subsection{Certification and Recognition}
\begin{itemize}
    \item Course completion certificate for successful participants
    \item LinkedIn skill endorsements
    \item Reference letters for exceptional performance
    \item Alumni network access for ongoing support
    \item Continuing education pathway recommendations
\end{itemize}

\subsection{Career Integration}
\begin{itemize}
    \item Professional portfolio development
    \item Technical interview preparation
    \item Industry mentor connections
    \item Job placement assistance
    \item Advanced certification pathways
\end{itemize}

\section{Course Evaluation}

\subsection{Continuous Feedback}
\begin{itemize}
    \item Daily exit tickets for immediate course adjustments
    \item Mid-course survey for major feedback incorporation
    \item Real-time polls during lectures for engagement monitoring
    \item Anonymous suggestion box for sensitive feedback
\end{itemize}

\subsection{Final Evaluation}
\begin{itemize}
    \item Comprehensive course evaluation survey
    \item Skills assessment pre/post comparison
    \item Long-term follow-up surveys (3, 6, 12 months)
    \item Employer feedback collection (where applicable)
    \item Course improvement recommendations
\end{itemize}

\section{Emergency Procedures}

\subsection{Technical Emergencies}
\begin{itemize}
    \item Backup hardware available for immediate replacement
    \item Alternative development environments (cloud-based, Docker)
    \item Recorded sessions for missed content due to technical issues
    \item One-on-one catch-up sessions for significant disruptions
\end{itemize}

\subsection{Health and Safety}
\begin{itemize}
    \item COVID-19 protocols as per institutional guidelines
    \item Ergonomic considerations for extended computer use
    \item Break schedules to prevent fatigue
    \item Accessibility accommodations as needed
    \item Emergency contact procedures
\end{itemize}

\section{Contact Information}

\subsection{Primary Contacts}
\begin{itemize}
    \item \textbf{Course Coordinator:} [Name, Email, Phone]
    \item \textbf{Lead Instructor:} [Name, Email, Office Hours]
    \item \textbf{Technical Support:} [Email, Response Time]
    \item \textbf{Administrative Support:} [Name, Email, Phone]
\end{itemize}

\subsection{Communication Channels}
\begin{itemize}
    \item \textbf{Primary:} Course GitHub organization discussions
    \item \textbf{Urgent:} Email to instructor and coordinator
    \item \textbf{Informal:} Course Slack workspace (optional)
    \item \textbf{Emergency:} Phone contact for critical issues
\end{itemize}

\vspace{1cm}

\begin{center}
\textbf{Welcome to C Programming for Post-Silicon Validation Engineers!}\\
\textit{We look forward to your transformation from programming novice to embedded validation expert.}
\end{center}

\end{document}

