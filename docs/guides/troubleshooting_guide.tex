\documentclass[11pt,a4paper]{article}
\usepackage[utf8]{inputenc}
\usepackage[T1]{fontenc}
\usepackage{geometry}
\usepackage{graphicx}
\usepackage{xcolor}
\usepackage{listings}
\usepackage{fancyhdr}
\usepackage{titlesec}
\usepackage{hyperref}
\usepackage{enumitem}
\usepackage{booktabs}
\usepackage{array}
\usepackage{tikz}

% Page setup
\geometry{margin=1in}
\setlength{\headheight}{14pt}
\pagestyle{fancy}
\fancyhf{}
\fancyhead[L]{Troubleshooting Guide}
\fancyhead[R]{\thepage}
\fancyfoot[C]{C Programming for Post-Silicon Validation Engineers}

% Colors
\definecolor{codeblue}{RGB}{0,102,204}
\definecolor{codegray}{RGB}{128,128,128}
\definecolor{codegreen}{RGB}{0,128,0}
\definecolor{backcolour}{RGB}{245,245,245}
\definecolor{warningred}{RGB}{220,20,60}
\definecolor{successgreen}{RGB}{34,139,34}

% Code listing style
\lstdefinestyle{bashstyle}{
    backgroundcolor=\color{backcolour},
    commentstyle=\color{codegreen},
    keywordstyle=\color{codeblue},
    numberstyle=\tiny\color{codegray},
    stringstyle=\color{red},
    basicstyle=\ttfamily\footnotesize,
    breakatwhitespace=false,
    breaklines=true,
    captionpos=b,
    keepspaces=true,
    showspaces=false,
    showstringspaces=false,
    showtabs=false,
    tabsize=2,
    frame=single
}

\lstset{style=bashstyle}

% Title formatting
\titleformat{\section}{\Large\bfseries\color{codeblue}}{\thesection}{1em}{}
\titleformat{\subsection}{\large\bfseries}{\thesubsection}{1em}{}

\hypersetup{
    colorlinks=true,
    linkcolor=codeblue,
    filecolor=magenta,
    urlcolor=cyan,
    pdftitle={Troubleshooting Guide},
    pdfauthor={Yahwista Salomo},
}

\begin{document}

% Header
\begin{center}
    {\Huge\bfseries\color{codeblue} TROUBLESHOOTING GUIDE}\\[0.5cm]
    {\Large C Programming for Post-Silicon Validation Engineers}\\[0.3cm]
    {\large Common Issues and Solutions}\\[0.2cm]
    {\normalsize Quick Reference for Students and Instructors}
\end{center}

\vspace{1cm}

\section{Quick Reference}

\subsection{Emergency Contacts}
\begin{itemize}
    \item \textbf{Lead Instructor:} yahwistasalomo.pv@oppstar.com.my
    \item \textbf{Technical Support:} yahwistasalomo.pv@oppstar.com.my
    \item \textbf{Course Slack:} \#c-programming-help
    \item \textbf{Emergency Phone:} [+1-XXX-XXX-XXXX] (hardware failures only)
\end{itemize}

\subsection{Most Common Issues}
\begin{enumerate}
    \item Compilation errors (syntax, missing headers)
    \item Cross-compilation toolchain problems
    \item MicroBlaze-V flashing and connection issues
    \item Git/GitHub workflow problems
    \item VS Code configuration issues
\end{enumerate}

\section{Compilation Issues}

\subsection{GCC Compilation Errors}

\subsubsection{Error: "gcc: command not found"}
\textbf{Symptoms:} Terminal/command prompt doesn't recognize gcc command.

\textbf{Causes:}
\begin{itemize}
    \item GCC not installed
    \item PATH environment variable not configured
    \item Using wrong terminal (Windows)
\end{itemize}

\textbf{Solutions:}
\begin{lstlisting}[language=bash]
# Check if GCC is installed
which gcc
gcc --version

# Windows: Use MSYS2 terminal or add to PATH
# Add C:\msys64\mingw64\bin to system PATH

# macOS: Install via Homebrew
brew install gcc

# Linux: Install via package manager
sudo apt install build-essential  # Ubuntu/Debian
sudo yum install gcc              # CentOS/RHEL
\end{lstlisting}

\subsubsection{Error: "fatal error: stdio.h: No such file or directory"}
\textbf{Symptoms:} Cannot find standard C headers.

\textbf{Causes:}
\begin{itemize}
    \item Incomplete GCC installation
    \item Missing C standard library headers
    \item Incorrect include paths
\end{itemize}

\textbf{Solutions:}
\begin{lstlisting}[language=bash]
# Linux: Install development headers
sudo apt install libc6-dev build-essential

# macOS: Install Xcode command line tools
xcode-select --install

# Windows: Reinstall MSYS2 with complete toolchain
pacman -S mingw-w64-x86_64-toolchain
\end{lstlisting}

\subsubsection{Error: ``undefined reference to 'function\_name''}
\textbf{Symptoms:} Linker cannot find function definitions.

\textbf{Causes:}
\begin{itemize}
    \item Function declared but not defined
    \item Missing source files in compilation
    \item Incorrect function signature
    \item Missing library links
\end{itemize}

\textbf{Solutions:}
\begin{lstlisting}[language=bash]
# Include all source files
gcc main.c validation.c register_access.c -o program

# Check function signatures match declarations
# Verify all functions are implemented

# For library functions, add appropriate flags
gcc program.c -lm -o program  # for math library
\end{lstlisting}

\subsection{Cross-Compilation Issues}

\subsubsection{Error: "arm-none-eabi-gcc: command not found"}
\textbf{Symptoms:} ARM cross-compiler not found.

\textbf{Solutions:}
\begin{lstlisting}[language=bash]
# Windows (MSYS2)
pacman -S mingw-w64-x86_64-arm-none-eabi-gcc

# macOS
brew install --cask gcc-arm-embedded
# or
brew tap ArmMbed/homebrew-formulae
brew install arm-none-eabi-gcc

# Linux
sudo apt install gcc-arm-none-eabi

# Verify installation
arm-none-eabi-gcc --version
\end{lstlisting}

\subsubsection{Error: "Could not find PICO\_SDK\_PATH"}
\textbf{Symptoms:} CMake cannot locate Pico SDK.

\textbf{Solutions:}
\begin{lstlisting}[language=bash]
# Check if environment variable is set
echo $PICO_SDK_PATH

# Set environment variable (Linux/macOS)
export PICO_SDK_PATH=~/Development/pico-sdk
echo 'export PICO_SDK_PATH=~/Development/pico-sdk' >> ~/.bashrc

# Windows: Set in System Environment Variables
# Variable: PICO_SDK_PATH
# Value: C:\dev\pico-sdk

# Verify SDK is properly cloned
ls $PICO_SDK_PATH
# Should show: cmake, docs, src, tools, etc.

# Re-clone if necessary
git clone --recursive https://github.com/raspberrypi/pico-sdk.git
\end{lstlisting}

\section{MicroBlaze-V Hardware Issues}

\subsection{Flashing and Connection Problems}

\subsubsection{Issue: "MicroBlaze-V not detected as USB drive"}
\textbf{Symptoms:} Pico doesn't appear as RPI-RP2 drive when connected.

\textbf{Solutions:}
\begin{enumerate}
    \item Hold BOOTSEL button while connecting USB cable
    \item Keep BOOTSEL pressed until drive appears
    \item Try different USB cable (data cable, not charge-only)
    \item Try different USB port
    \item Check Device Manager (Windows) for unrecognized devices
\end{enumerate}

\subsubsection{Issue: "Program doesn't run after flashing UF2"}
\textbf{Symptoms:} UF2 copies successfully but program doesn't execute.

\textbf{Diagnostic Steps:}
\begin{lstlisting}[language=bash]
# Check UF2 file size (should be > 0 bytes)
ls -la program.uf2

# Verify build completed successfully
make clean
make

# Check for build errors in output
# Look for "failed" or "error" messages

# Try a simple blink program first
\end{lstlisting}

\textbf{Solutions:}
\begin{itemize}
    \item Rebuild project from clean state
    \item Verify CMakeLists.txt configuration
    \item Check that pico\_add\_extra\_outputs() is called
    \item Try flashing a known-good UF2 file
    \item Reset Pico by unplugging and reconnecting
\end{itemize}

\subsection{Serial Communication Issues}

\subsubsection{Issue: "No serial output from MicroBlaze-V"}
\textbf{Symptoms:} Program runs but no printf output visible.

\textbf{Solutions:}
\begin{lstlisting}[language=bash]
# Verify UART stdio is enabled in your platform configuration
# For MicroBlaze-V, configure UART peripheral

# Add stdio initialization in code
#include <stdio.h>
// Initialize UART for your specific MicroBlaze-V configuration
// uart_init(UART_ID, BAUD_RATE);
// stdio_set_driver_enabled(&stdio_uart, true);
\end{lstlisting}

\textbf{Terminal Setup:}
\begin{itemize}
    \item \textbf{Windows:} Use PuTTY, Tera Term, or Windows Terminal
    \item \textbf{macOS:} Use screen, minicom, or Serial app
    \item \textbf{Linux:} Use screen, minicom, or picocom
\end{itemize}

\begin{lstlisting}[language=bash]
# Find the correct serial port
# Windows: Check Device Manager for COM port
# macOS/Linux:
ls /dev/tty*  # Look for /dev/ttyUSB0, /dev/ttyACM0, or similar

# Connect using screen (macOS/Linux)
screen /dev/ttyUSB0 115200

# Connect using minicom
minicom -D /dev/ttyUSB0 -b 115200
\end{lstlisting}

\section{Development Environment Issues}

\subsection{VS Code Problems}

\subsubsection{Issue: "IntelliSense not working for C files"}
\textbf{Symptoms:} No syntax highlighting, autocomplete, or error detection.

\textbf{Solutions:}
\begin{enumerate}
    \item Install C/C++ extension by Microsoft
    \item Configure compiler path in settings
    \item Create or update \texttt{c\_cpp\_properties.json}
    \item Reload VS Code window
\end{enumerate}

\begin{lstlisting}
// .vscode/c_cpp_properties.json
{
    "configurations": [
        {
            "name": "Linux",
            "includePath": [
                "${workspaceFolder}/**",
                "${env:MICROBLAZE_SDK_PATH}/src/**"
            ],
            "defines": [],
            "compilerPath": "/usr/bin/gcc",
            "cStandard": "c11",
            "cppStandard": "c++17",
            "intelliSenseMode": "linux-gcc-x64"
        }
    ],
    "version": 4
}
\end{lstlisting}

\subsubsection{Issue: "CMake extension not detecting project"}
\textbf{Symptoms:} CMake tools don't recognize CMakeLists.txt.

\textbf{Solutions:}
\begin{enumerate}
    \item Install CMake Tools extension
    \item Open Command Palette (Ctrl+Shift+P)
    \item Run "CMake: Configure"
    \item Select appropriate compiler kit
    \item Check CMakeLists.txt syntax
\end{enumerate}

\subsection{Git and GitHub Issues}

\subsubsection{Issue: "Permission denied (publickey)" when pushing to GitHub}
\textbf{Symptoms:} Cannot push commits to GitHub repository.

\textbf{Solutions:}
\begin{lstlisting}[language=bash]
# Check if SSH key exists
ls ~/.ssh/id_*

# Generate new SSH key if needed
ssh-keygen -t ed25519 -C "yahwistasalomo.pv@oppstar.com.my"

# Add key to SSH agent
eval "$(ssh-agent -s)"
ssh-add ~/.ssh/id_ed25519

# Copy public key and add to GitHub
cat ~/.ssh/id_ed25519.pub

# Test SSH connection
ssh -T git@github.com

# Alternative: Use HTTPS instead of SSH
git remote set-url origin https://github.com/username/repo.git
\end{lstlisting}

\subsubsection{Issue: "Merge conflicts in Git"}
\textbf{Symptoms:} Cannot merge or pull due to conflicting changes.

\textbf{Solutions:}
\begin{lstlisting}[language=bash]
# Check status
git status

# View conflicted files
git diff

# Edit files to resolve conflicts
# Look for <<<<<<< ======= >>>>>>> markers
# Choose which version to keep

# Stage resolved files
git add resolved_file.c

# Complete merge
git commit -m "Resolve merge conflict"

# Alternative: Abort merge and start over
git merge --abort
\end{lstlisting}

\section{CMake Build System Issues}

\subsection{Common CMake Errors}

\subsubsection{Error: "CMake version too old"}
\textbf{Symptoms:} "CMake 3.13 or higher is required"

\textbf{Solutions:}
\begin{lstlisting}[language=bash]
# Check current version
cmake --version

# Update CMake
# Ubuntu/Debian
sudo apt remove cmake
sudo snap install cmake --classic

# macOS
brew upgrade cmake

# Windows (download from cmake.org)
# Or use chocolatey: choco upgrade cmake
\end{lstlisting}

\subsubsection{Error: ``No CMAKE\_C\_COMPILER could be found''}
\textbf{Symptoms:} CMake cannot find C compiler.

\textbf{Solutions:}
\begin{lstlisting}[language=bash]
# Specify compiler explicitly
cmake -DCMAKE_C_COMPILER=gcc ..

# For cross-compilation
cmake -DCMAKE_C_COMPILER=riscv64-unknown-elf-gcc ..

# Clear CMake cache and reconfigure
rm -rf build/
mkdir build && cd build
cmake ..
\end{lstlisting}

\section{Runtime and Logic Errors}

\subsection{Segmentation Faults}

\subsubsection{Issue: "Segmentation fault (core dumped)"}
\textbf{Symptoms:} Program crashes with segfault.

\textbf{Common Causes:}
\begin{itemize}
    \item Dereferencing NULL pointers
    \item Array bounds violations
    \item Stack overflow (infinite recursion)
    \item Use after free
\end{itemize}

\textbf{Debugging with GDB:}
\begin{lstlisting}[language=bash]
# Compile with debug symbols
gcc -g -Wall program.c -o program

# Run with GDB
gdb ./program
(gdb) run
# Program crashes
(gdb) backtrace
(gdb) print variable_name
(gdb) list
\end{lstlisting}

\textbf{Prevention:}
\begin{lstlisting}[language=c]
// Always check pointers
if (ptr != NULL) {
    *ptr = value;
}

// Check array bounds
if (index >= 0 && index < array_size) {
    array[index] = value;
}

// Initialize pointers
int *ptr = NULL;
\end{lstlisting}

\subsection{Memory Issues}

\subsubsection{Issue: "Memory leaks"}
\textbf{Symptoms:} Program uses increasing amounts of memory.

\textbf{Detection:}
\begin{lstlisting}[language=bash]
# Use Valgrind (Linux/macOS)
valgrind --leak-check=full ./program

# Use AddressSanitizer
gcc -fsanitize=address -g program.c -o program
./program
\end{lstlisting}

\textbf{Prevention:}
\begin{lstlisting}[language=c]
// Always free allocated memory
char *buffer = malloc(100);
if (buffer != NULL) {
    // use buffer
    free(buffer);
    buffer = NULL;  // Prevent double-free
}

// Use static allocation when possible
char buffer[100];  // No need to free
\end{lstlisting}

\section{Performance Issues}

\subsection{Slow Compilation}

\textbf{Symptoms:} Build takes very long time.

\textbf{Solutions:}
\begin{lstlisting}[language=bash]
# Use parallel compilation
make -j4  # Use 4 cores

# For CMake
cmake --build . --parallel 4

# Clean unnecessary files
make clean

# Use ccache for faster rebuilds
sudo apt install ccache
export CC="ccache gcc"
\end{lstlisting}

\subsection{Slow Program Execution}

\textbf{Symptoms:} Program runs slower than expected.

\textbf{Optimization:}
\begin{lstlisting}[language=bash]
# Compile with optimization
gcc -O2 program.c -o program

# For embedded targets
arm-none-eabi-gcc -O2 -DNDEBUG program.c

# Profile with gprof
gcc -pg program.c -o program
./program
gprof program gmon.out > analysis.txt
\end{lstlisting}

\section{Platform-Specific Issues}

\subsection{Windows-Specific Problems}

\subsubsection{Issue: "Windows Defender blocking compilation"}
\textbf{Symptoms:} Antivirus deletes compiled executables.

\textbf{Solutions:}
\begin{itemize}
    \item Add development folder to Windows Defender exclusions
    \item Temporarily disable real-time protection during development
    \item Use Windows Security → Virus \& threat protection → Exclusions
\end{itemize}

\subsubsection{Issue: "Long path names causing issues"}
\textbf{Symptoms:} Build fails with path-related errors.

\textbf{Solutions:}
\begin{itemize}
    \item Use shorter directory names
    \item Enable long path support in Windows 10/11
    \item Move project closer to root directory (e.g., C:\textbackslash dev\textbackslash)
\end{itemize}

\subsection{macOS-Specific Problems}

\subsubsection{Issue: "Operation not permitted" errors}
\textbf{Symptoms:} Cannot write to certain directories.

\textbf{Solutions:}
\begin{itemize}
    \item Use user directories instead of system directories
    \item Grant Terminal full disk access in System Preferences
    \item Use sudo only when necessary
\end{itemize}

\subsection{Linux-Specific Problems}

\subsubsection{Issue: "Permission denied" for USB devices}
\textbf{Symptoms:} Cannot access MicroBlaze-V or serial ports.

\textbf{Solutions:}
\begin{lstlisting}[language=bash]
# Add user to dialout group
sudo usermod -a -G dialout $USER

# Set udev rules for MicroBlaze-V
sudo tee /etc/udev/rules.d/99-pico.rules << 'EOF'
SUBSYSTEM=="usb", ATTRS{idVendor}=="2e8a", MODE="0666"
EOF

# Reload udev rules
sudo udevadm control --reload-rules
sudo udevadm trigger

# Log out and back in for group changes
\end{lstlisting}

\section{Emergency Procedures}

\subsection{Complete Environment Reset}

If your development environment is completely broken:

\begin{enumerate}
    \item \textbf{Backup your code:} Copy all source files to a safe location
    \item \textbf{Document the problem:} Note error messages and steps that led to the issue
    \item \textbf{Clean reinstall:} Follow the setup guide from scratch
    \item \textbf{Restore code:} Copy your source files back
    \item \textbf{Test incrementally:} Verify each tool works before proceeding
\end{enumerate}

\subsection{Hardware Failure}

If your MicroBlaze-V board stops working:

\begin{enumerate}
    \item Try different USB cable and port
    \item Test with known-good UF2 file (blink example)
    \item Check for physical damage
    \item Request replacement board from instructor
    \item Use QEMU RISC-V virt machine emulation as temporary fallback
\end{enumerate}

\subsection{Data Recovery}

If you lose your work:

\begin{enumerate}
    \item Check Git history: \texttt{git log --oneline}
    \item Look for auto-saved files in VS Code
    \item Check temporary directories
    \item Restore from GitHub if you've been pushing regularly
    \item Ask teammates for their copies of shared code
\end{enumerate}

\section{Prevention and Best Practices}

\subsection{Avoiding Common Issues}

\begin{itemize}
    \item \textbf{Commit frequently:} Save your work to Git every 30 minutes
    \item \textbf{Test incrementally:} Compile and test after small changes
    \item \textbf{Read error messages:} Don't ignore compiler warnings
    \item \textbf{Keep backups:} Push to GitHub regularly
    \item \textbf{Document problems:} Keep notes on issues and solutions
\end{itemize}

\subsection{Good Development Habits}

\begin{itemize}
    \item Start with simple programs and add complexity gradually
    \item Use meaningful variable and function names
    \item Add comments to explain complex logic
    \item Test edge cases and error conditions
    \item Keep your development environment updated
\end{itemize}

\section{Getting Help}

\subsection{Self-Help Resources}

\begin{enumerate}
    \item \textbf{Read error messages carefully:} They often contain the solution
    \item \textbf{Search online:} Stack Overflow, GitHub issues, documentation
    \item \textbf{Check course materials:} Slides, examples, and documentation
    \item \textbf{Use debugging tools:} GDB, print statements, LED indicators
\end{enumerate}

\subsection{When to Ask for Help}

Ask for help when:
\begin{itemize}
    \item You've spent more than 30 minutes on a single error
    \item Error messages are unclear or contradictory
    \item Hardware appears to be malfunctioning
    \item You need clarification on requirements
    \item You're completely stuck and don't know where to start
\end{itemize}

\subsection{How to Ask for Help Effectively}

When requesting help, provide:
\begin{itemize}
    \item \textbf{Complete error message:} Copy and paste the full error
    \item \textbf{Context:} What were you trying to do?
    \item \textbf{Environment:} OS, compiler version, etc.
    \item \textbf{Code snippet:} Minimal example that reproduces the problem
    \item \textbf{Steps tried:} What solutions have you already attempted?
\end{itemize}

\vspace{1cm}

\begin{center}
\textbf{Remember: Every expert was once a beginner!}\\
\textit{Troubleshooting is a valuable skill that improves with practice.}\\
\textbf{Don't hesitate to ask for help when you need it.}
\end{center}

\end{document}

