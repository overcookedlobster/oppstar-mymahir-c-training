\documentclass[11pt,a4paper]{article}
\usepackage[utf8]{inputenc}
\usepackage[T1]{fontenc}
\usepackage{geometry}
\usepackage{graphicx}
\usepackage{xcolor}
\usepackage{listings}
\usepackage{fancyhdr}
\usepackage{titlesec}
\usepackage{hyperref}
\usepackage{enumitem}
\usepackage{booktabs}
\usepackage{array}
\usepackage{tikz}
\usepackage{amssymb}

% Page setup
\geometry{margin=1in}
\setlength{\headheight}{14pt}
\pagestyle{fancy}
\fancyhf{}
\fancyhead[L]{MicroBlaze-V Development Environment Setup Guide}
\fancyhead[R]{\thepage}
\fancyfoot[C]{C Programming for Post-Silicon Validation Engineers}

% Colors
\definecolor{codeblue}{RGB}{0,102,204}
\definecolor{codegray}{RGB}{128,128,128}
\definecolor{codegreen}{RGB}{0,128,0}
\definecolor{backcolour}{RGB}{245,245,245}
\definecolor{warningred}{RGB}{220,20,60}
\definecolor{successgreen}{RGB}{34,139,34}

% Code listing style
\lstdefinestyle{bashstyle}{
    backgroundcolor=\color{backcolour},
    commentstyle=\color{codegreen},
    keywordstyle=\color{codeblue},
    numberstyle=\tiny\color{codegray},
    stringstyle=\color{red},
    basicstyle=\ttfamily\footnotesize,
    breakatwhitespace=false,
    breaklines=true,
    captionpos=b,
    keepspaces=true,
    showspaces=false,
    showstringspaces=false,
    showtabs=false,
    tabsize=2,
    frame=single
}

\lstset{style=bashstyle}

% Title formatting
\titleformat{\section}{\Large\bfseries\color{codeblue}}{\thesection}{1em}{}
\titleformat{\subsection}{\large\bfseries}{\thesubsection}{1em}{}

\hypersetup{
    colorlinks=true,
    linkcolor=codeblue,
    filecolor=magenta,
    urlcolor=cyan,
    pdftitle={Development Environment Setup Guide},
    pdfauthor={Course Instructor},
}

\begin{document}

% Header
\begin{center}
    {\Huge\bfseries\color{codeblue} DEVELOPMENT ENVIRONMENT}\\[0.3cm]
    {\Huge\bfseries\color{codeblue} SETUP GUIDE}\\[0.5cm]
    {\Large C Programming for Post-Silicon Validation Engineers}\\[0.3cm]
    {\large Complete Installation and Configuration Instructions}\\[0.2cm]
    {\normalsize Version 1.0 - 2025}
\end{center}

\vspace{1cm}

\section{Overview}

This guide provides step-by-step instructions for setting up your development environment for the 6-day C programming course. You must complete this setup \textbf{one week before} the course begins.

\subsection{What You'll Install}
\begin{itemize}
    \item \textbf{GCC Compiler:} Native C compiler for desktop development
    \item \textbf{RISC-V Cross-Compiler:} For MicroBlaze-V embedded development
    \item \textbf{CMake:} Build system for managing complex projects
    \item \textbf{Git:} Version control system
    \item \textbf{VS Code:} Primary development environment
    \item \textbf{MicroBlaze-V SDK:} Embedded development framework
    \item \textbf{GDB:} Debugger for troubleshooting code
    \item \textbf{Python 3:} For build scripts and utilities
\end{itemize}

\subsection{Time Required}
\begin{itemize}
    \item \textbf{Windows:} 2-3 hours
    \item \textbf{macOS:} 1-2 hours
    \item \textbf{Linux:} 30-60 minutes
\end{itemize}

\section{Pre-Installation Checklist}

\subsection{System Requirements}
\begin{itemize}
    \item \textbf{Operating System:} Windows 10/11, macOS 10.15+, or Linux (Ubuntu 20.04+ recommended)
    \item \textbf{RAM:} Minimum 4GB, recommended 8GB+
    \item \textbf{Storage:} At least 5GB free space
    \item \textbf{Internet:} Stable connection for downloads
    \item \textbf{Administrator Access:} Required for software installation
\end{itemize}

\subsection{Before You Begin}
\begin{itemize}
    \item Close all unnecessary applications
    \item Disable antivirus temporarily (if it interferes with installation)
    \item Have your GitHub account ready (create one at \url{https://github.com} if needed)
    \item Ensure you have administrator/sudo privileges
\end{itemize}

\section{Windows Installation}

\subsection{Step 1: Install MSYS2}
MSYS2 provides a Unix-like environment on Windows with package management.

\begin{enumerate}
    \item Download MSYS2 from \url{https://www.msys2.org/}
    \item Run the installer and follow the default installation path: \texttt{C:\textbackslash msys64}
    \item After installation, open MSYS2 terminal
    \item Update the package database:
\end{enumerate}

\begin{lstlisting}[language=bash]
pacman -Syu
\end{lstlisting}

\textbf{Note:} The terminal may close after the first update. Reopen it and run the command again.

\subsection{Step 2: Install Development Tools}
Install the essential development tools through MSYS2:

\begin{lstlisting}[language=bash]
# Install base development tools
pacman -S mingw-w64-x86_64-gcc
pacman -S mingw-w64-x86_64-gdb
pacman -S mingw-w64-x86_64-cmake
pacman -S git
pacman -S make

# Install RISC-V cross-compiler for MicroBlaze-V
pacman -S mingw-w64-x86_64-riscv64-unknown-elf-gcc
pacman -S mingw-w64-x86_64-riscv64-unknown-elf-gdb

# Install Python
pacman -S mingw-w64-x86_64-python
pacman -S mingw-w64-x86_64-python-pip
\end{lstlisting}

\subsection{Step 3: Configure Environment Variables}
Add MSYS2 tools to your Windows PATH:

\begin{enumerate}
    \item Open Windows Settings → System → About → Advanced System Settings
    \item Click "Environment Variables"
    \item Under "System Variables," find and edit "Path"
    \item Add these paths (adjust if you installed MSYS2 elsewhere):
    \begin{itemize}
        \item \texttt{C:\textbackslash msys64\textbackslash mingw64\textbackslash bin}
        \item \texttt{C:\textbackslash msys64\textbackslash usr\textbackslash bin}
    \end{itemize}
    \item Click OK to save changes
    \item Restart your computer
\end{enumerate}

\subsection{Step 4: Install VS Code}
\begin{enumerate}
    \item Download VS Code from \url{https://code.visualstudio.com/}
    \item Run the installer with default settings
    \item Launch VS Code and install these extensions:
    \begin{itemize}
        \item C/C++ (Microsoft)
        \item CMake Tools (Microsoft)
        \item GitLens (GitKraken)
        \item RISC-V (Zihao Yu)
    \end{itemize}
\end{enumerate}

\subsection{Step 5: Install MicroBlaze-V SDK}
\begin{enumerate}
    \item Open Command Prompt or PowerShell as Administrator
    \item Navigate to your development directory (e.g., \texttt{C:\textbackslash dev})
    \item Clone the MicroBlaze-V SDK:
\end{enumerate}

\begin{lstlisting}[language=bash]
git clone --recursive https://github.com/Xilinx/embeddedsw.git
\end{lstlisting}

\begin{enumerate}
    \setcounter{enumi}{3}
    \item Set the MICROBLAZE\_SDK\_PATH environment variable:
    \begin{itemize}
        \item Go to Environment Variables (as in Step 3)
        \item Create a new System Variable:
        \item Name: \texttt{MICROBLAZE\_SDK\_PATH}
        \item Value: \texttt{C:\textbackslash dev\textbackslash embeddedsw} (adjust path as needed)
    \end{itemize}
\end{enumerate}

\section{macOS Installation}

\subsection{Step 1: Install Homebrew}
Homebrew is the package manager for macOS:

\begin{lstlisting}[language=bash]
/bin/bash -c "$(curl -fsSL https://raw.githubusercontent.com/Homebrew/install/HEAD/install.sh)"
\end{lstlisting}

Follow the on-screen instructions to add Homebrew to your PATH.

\subsection{Step 2: Install Development Tools}
\begin{lstlisting}[language=bash]
# Install essential tools
brew install gcc
brew install cmake
brew install git
brew install python3

# Install RISC-V cross-compiler for MicroBlaze-V
brew install riscv-gnu-toolchain
\end{lstlisting}

\subsection{Step 3: Install VS Code}
\begin{lstlisting}[language=bash]
brew install --cask visual-studio-code
\end{lstlisting}

Or download directly from \url{https://code.visualstudio.com/}

Install the same extensions as listed in the Windows section.

\subsection{Step 4: Install MicroBlaze-V SDK}
\begin{lstlisting}[language=bash]
# Navigate to your development directory
cd ~/Development  # or wherever you keep projects
mkdir -p ~/Development  # create if it doesn't exist

# Clone the MicroBlaze-V SDK
git clone --recursive https://github.com/Xilinx/embeddedsw.git

# Set environment variable
echo 'export MICROBLAZE_SDK_PATH=~/Development/embeddedsw' >> ~/.zshrc
# For bash users, use ~/.bash_profile instead of ~/.zshrc

# Reload your shell configuration
source ~/.zshrc  # or source ~/.bash_profile
\end{lstlisting}

\section{Linux Installation (Ubuntu/Debian)}

\subsection{Step 1: Update System}
\begin{lstlisting}[language=bash]
sudo apt update
sudo apt upgrade -y
\end{lstlisting}

\subsection{Step 2: Install Development Tools}
\begin{lstlisting}[language=bash]
# Install essential development tools
sudo apt install -y build-essential gcc gdb cmake git python3 python3-pip

# Install RISC-V cross-compiler for MicroBlaze-V
sudo apt install -y gcc-riscv64-unknown-elf gdb-multiarch

# Install additional utilities
sudo apt install -y curl wget unzip
\end{lstlisting}

\subsection{Step 3: Install VS Code}
\textbf{Option 1: Using Snap (Recommended)}
\begin{lstlisting}[language=bash]
sudo snap install code --classic
\end{lstlisting}

\textbf{Option 2: Using APT Repository}
\begin{lstlisting}[language=bash]
# Add Microsoft GPG key and repository
wget -qO- https://packages.microsoft.com/keys/microsoft.asc | gpg --dearmor > packages.microsoft.gpg
sudo install -o root -g root -m 644 packages.microsoft.gpg /etc/apt/trusted.gpg.d/
sudo sh -c 'echo "deb [arch=amd64,arm64,armhf signed-by=/etc/apt/trusted.gpg.d/packages.microsoft.gpg] https://packages.microsoft.com/repos/code stable main" > /etc/apt/sources.list.d/vscode.list'

# Install VS Code
sudo apt update
sudo apt install code
\end{lstlisting}

\subsection{Step 4: Install MicroBlaze-V SDK}
\begin{lstlisting}[language=bash]
# Navigate to your development directory
cd ~/Development
mkdir -p ~/Development  # create if it doesn't exist

# Clone the MicroBlaze-V SDK
git clone --recursive https://github.com/Xilinx/embeddedsw.git

# Set environment variable
echo 'export MICROBLAZE_SDK_PATH=~/Development/embeddedsw' >> ~/.bashrc
source ~/.bashrc
\end{lstlisting}

\section{Verification and Testing}

\subsection{Step 1: Verify Tool Installation}
Open a terminal/command prompt and run these commands to verify installation:

\begin{lstlisting}[language=bash]
# Check GCC
gcc --version
# Expected: gcc (GCC) 11.x.x or newer

# Check RISC-V cross-compiler
riscv64-unknown-elf-gcc --version
# Expected: riscv64-unknown-elf-gcc (GCC) 10.x.x or newer

# Check CMake
cmake --version
# Expected: cmake version 3.13.x or newer

# Check Git
git --version
# Expected: git version 2.x.x

# Check Python
python3 --version
# Expected: Python 3.8.x or newer

# Check GDB
gdb --version
# Expected: GNU gdb (GDB) 10.x.x or newer

# Check MicroBlaze-V SDK
echo $MICROBLAZE_SDK_PATH
# Expected: Path to your embeddedsw directory
\end{lstlisting}

\subsection{Step 2: Test Basic Compilation}
Create a test program to verify your setup:

\begin{lstlisting}[language=bash]
# Create a test directory
mkdir ~/test-setup
cd ~/test-setup

# Create a simple C program
cat > hello.c << 'EOF'
#include <stdio.h>

int main() {
    printf("Hello, Validation World!\n");
    printf("GCC is working correctly!\n");
    return 0;
}
EOF

# Compile and run the program
gcc -Wall -g hello.c -o hello
./hello
\end{lstlisting}

\textbf{Expected Output:}
\begin{verbatim}
Hello, Validation World!
GCC is working correctly!
\end{verbatim}

\subsection{Step 3: Test Cross-Compilation}
Test ARM cross-compilation with a simple Pico program:

\begin{lstlisting}[language=bash]
# Create a simple Pico program
cat > pico_test.c << 'EOF'
#include "pico/stdlib.h"

int main() {
    stdio_init_all();
    printf("Pico SDK is working!\n");
    return 0;
}
EOF

# Create a basic CMakeLists.txt
cat > CMakeLists.txt << 'EOF'
cmake_minimum_required(VERSION 3.13)
include($ENV{PICO_SDK_PATH}/external/pico_sdk_import.cmake)
project(pico_test C CXX ASM)
set(CMAKE_C_STANDARD 11)
pico_sdk_init()
add_executable(pico_test pico_test.c)
target_link_libraries(pico_test pico_stdlib)
pico_enable_stdio_usb(pico_test 1)
pico_add_extra_outputs(pico_test)
EOF

# Build the project
mkdir build
cd build
cmake ..
make
\end{lstlisting}

If successful, you should see a \texttt{pico\_test.uf2} file in the build directory.

\section{VS Code Configuration}

\subsection{Configure C/C++ Extension}
\begin{enumerate}
    \item Open VS Code
    \item Press \texttt{Ctrl+Shift+P} (or \texttt{Cmd+Shift+P} on macOS)
    \item Type "C/C++: Edit Configurations (UI)"
    \item Set the following:
    \begin{itemize}
        \item \textbf{Compiler path:} Path to your gcc executable
        \item \textbf{IntelliSense mode:} gcc-x64 (or appropriate for your system)
        \item \textbf{C standard:} c11
        \item \textbf{C++ standard:} c++17
    \end{itemize}
\end{enumerate}

\subsection{Configure CMake Extension}
\begin{enumerate}
    \item Open the Command Palette (\texttt{Ctrl+Shift+P})
    \item Type "CMake: Configure"
    \item Select your compiler when prompted
    \item The extension should automatically detect your CMake installation
\end{enumerate}

\subsection{Recommended VS Code Settings}
Create a \texttt{.vscode/settings.json} file in your project directory:

\begin{lstlisting}
{
    "C_Cpp.default.cStandard": "c11",
    "C_Cpp.default.cppStandard": "c++17",
    "C_Cpp.default.compilerPath": "/usr/bin/gcc",
    "cmake.configureOnOpen": true,
    "files.associations": {
        "*.h": "c",
        "*.c": "c"
    },
    "editor.tabSize": 4,
    "editor.insertSpaces": true,
    "editor.detectIndentation": false
}
\end{lstlisting}

\section{GitHub Setup}

\subsection{Configure Git}
Set up your Git identity:

\begin{lstlisting}[language=bash]
git config --global user.name "Your Full Name"
git config --global user.email "your.email@example.com"
git config --global init.defaultBranch main
\end{lstlisting}

\subsection{SSH Key Setup (Recommended)}
\begin{lstlisting}[language=bash]
# Generate SSH key
ssh-keygen -t ed25519 -C "your.email@example.com"

# Start SSH agent
eval "$(ssh-agent -s)"

# Add key to agent
ssh-add ~/.ssh/id_ed25519

# Copy public key to clipboard
# On Linux/macOS:
cat ~/.ssh/id_ed25519.pub
# On Windows (in Git Bash):
clip < ~/.ssh/id_ed25519.pub
\end{lstlisting}

\begin{enumerate}
    \item Go to GitHub.com → Settings → SSH and GPG keys
    \item Click "New SSH key"
    \item Paste your public key and save
    \item Test the connection: \texttt{ssh -T git@github.com}
\end{enumerate}

\section{Troubleshooting}

\subsection{Common Issues and Solutions}

\subsubsection{Windows: "gcc not found" or "command not recognized"}
\textbf{Problem:} PATH environment variable not set correctly.

\textbf{Solution:}
\begin{enumerate}
    \item Verify MSYS2 installation paths are in your system PATH
    \item Restart your computer after changing PATH
    \item Open a new command prompt/PowerShell window
    \item Try running commands from MSYS2 terminal instead
\end{enumerate}

\subsubsection{macOS: "xcrun: error: invalid active developer path"}
\textbf{Problem:} Xcode command line tools not installed.

\textbf{Solution:}
\begin{lstlisting}[language=bash]
xcode-select --install
\end{lstlisting}

\subsubsection{Linux: "arm-none-eabi-gcc: command not found"}
\textbf{Problem:} ARM toolchain not installed or not in PATH.

\textbf{Solution:}
\begin{lstlisting}[language=bash]
# Try alternative installation method
sudo apt remove gcc-arm-none-eabi
wget https://developer.arm.com/-/media/Files/downloads/gnu-rm/10.3-2021.10/gcc-arm-none-eabi-10.3-2021.10-x86_64-linux.tar.bz2
tar -xjf gcc-arm-none-eabi-10.3-2021.10-x86_64-linux.tar.bz2
sudo mv gcc-arm-none-eabi-10.3-2021.10 /opt/
echo 'export PATH=/opt/gcc-arm-none-eabi-10.3-2021.10/bin:$PATH' >> ~/.bashrc
source ~/.bashrc
\end{lstlisting}

\subsubsection{CMake: "Could not find PICO\_SDK\_PATH"}
\textbf{Problem:} Environment variable not set or not accessible.

\textbf{Solution:}
\begin{enumerate}
    \item Verify the Pico SDK was cloned with \texttt{--recursive} flag
    \item Check that PICO\_SDK\_PATH points to the correct directory
    \item Restart your terminal/IDE after setting the environment variable
    \item On Windows, ensure you set it as a System variable, not User variable
\end{enumerate}

\subsubsection{VS Code: IntelliSense not working}
\textbf{Problem:} C/C++ extension not configured properly.

\textbf{Solution:}
\begin{enumerate}
    \item Press \texttt{Ctrl+Shift+P} and run "C/C++: Reset IntelliSense Database"
    \item Check that compiler path is correct in \texttt{c\_cpp\_properties.json}
    \item Ensure include paths are set correctly
    \item Reload VS Code window
\end{enumerate}

\subsection{Getting Help}

If you encounter issues not covered here:

\begin{enumerate}
    \item Check the course discussion forum/Slack channel
    \item Search for error messages online (Stack Overflow, GitHub issues)
    \item Contact the course instructors with:
    \begin{itemize}
        \item Your operating system and version
        \item Complete error messages
        \item Steps you've already tried
        \item Screenshots if applicable
    \end{itemize}
    \item Attend pre-course office hours (if available)
\end{enumerate}

\section{Final Verification Checklist}

Before the course begins, ensure you can complete all these tasks:

\begin{itemize}
    \item[$\square$] All tools show correct version numbers when checked
    \item[$\square$] Simple C program compiles and runs successfully
    \item[$\square$] ARM cross-compilation produces .uf2 file for Pico
    \item[$\square$] VS Code opens and recognizes C files with syntax highlighting
    \item[$\square$] Git is configured with your name and email
    \item[$\square$] GitHub SSH key is set up and working
    \item[$\square$] CMake can configure a simple project
    \item[$\square$] GDB can be launched (even if you don't know how to use it yet)
    \item[$\square$] PICO\_SDK\_PATH environment variable is set correctly
    \item[$\square$] You can create, edit, and save files in your development directory
\end{itemize}

\section{What's Next?}

\subsection{Pre-Course Preparation}
\begin{enumerate}
    \item Complete the Git tutorial: \url{https://learngitbranching.js.org/}
    \item Watch CS50 C introduction videos (first 2 lectures)
    \item Read the Pico SDK "Getting Started" guide
    \item Familiarize yourself with VS Code interface
    \item Join the course communication channels
\end{enumerate}

\subsection{Day 1 Preparation}
\begin{itemize}
    \item Bring your laptop with all tools installed and verified
    \item Have your GitHub account credentials ready
    \item Ensure your development environment is working
    \item Bring a notebook for taking notes
    \item Come ready to learn and collaborate!
\end{itemize}

\vspace{1cm}

\begin{center}
\textbf{Congratulations on completing the setup!}\\
\textit{You're now ready to begin your journey into C programming and embedded validation.}\\
\textbf{See you on Day 1!}
\end{center}

\end{document}

